\documentclass{../notatki}

\title{Algebra liniowa}

\begin{document}

\tableofcontents

\section{Uporządkowana para liczb}

$$
(a, b) = \{ \{a\}, \{a, b\} \}
$$

\section{Grupa}

Grupa to uporządkowana para $G(A, \circ)$, gdzie $A$ to zbiór, a $\circ$ to działanie spełniające następujące warunki:

\begin{itemize}
    \item Zachodzi łączność działania $\forall a, b, c \in A: a \circ (b \circ c) = (a \circ b) \circ c$
    \item Istnieje element neutralny $e \in A: \forall a \in A: a \circ e = e \circ a = a$
    \item Dla każdego elementu istnieje element odwrotny $\forall a \in A: \exists a^{-1} \in A: a \circ a^{-1} = a^{-1} \circ a = e$
\end{itemize}

\subsection{Grupa abelowa}

Grupa abelowa, to specjalny rodzaj grupy w którym spełniony jest dodatkowy warunek:

\begin{itemize}
    \item Grupa jest przemienna $\forall a, b \in A: a \circ b = b \circ a$
\end{itemize}

\subsection{Przykłady grup}

$$
G(\mathbb{Z}, +), G(\mathbb{Q}, +), G(\mathbb{R}, +), G(\mathbb{C}, +)
$$

\section{Pierścień}

Pierścień to uporządkowana trójka $R(A, +, \cdot)$, gdzie $A$ to zbiór, a $+$ i $\cdot$ to działania spełniające następujące warunki:

\begin{itemize}
    \item $(A, +)$ jest grupą abelową
    \item $+$ i $\cdot$ są są wewnętrzne dla $A$
    \item Dla każdego $a, b, c \in A$ zachodzi rozdzielność mnożenia względem dodawania: $a \cdot (b + c) = a \cdot b + a \cdot c$ oraz $(a + b) \cdot c = a \cdot c + b \cdot c$
    \item Istnieje element neutralny mnożenia $1 \in A: \forall a \in A: a \cdot 1 = 1 \cdot a = a$
\end{itemize}

\subsection{Pierścień z jedynką}

Pierścień z jedynką to pierścień, w którym istnieje element neutralny mnożenia oraz $A \ne \emptyset$

\subsection{Pierścień przemienny}

Pierścień przemienny to pierścień, w którym mnożenie jest przemienna

\section{Ciało}

Ciało $C(K, +, \cdot)$ to pierścień przemienny z jedynką, oraz $(K \setminus \{0\}, \cdot)$ jest grupą

\section{Homomorfizmy}

Homomorfizmy to odwzorowania $f: A \rightarrow B$, jeśli $A$ i $B$ spełniają dodatkowe warunki..

\subsection{Homomorfizmy grupy}

Jeśli $(A, +_A)$ i $(B, +_B)$ to grupy oraz 

$$
\forall_{a \in A, b \in B} f(a +_A b) = f(a) +_B f(b) 
$$

\subsection{Homomorfizmy pierścieni}

Jeśli $(A, +_A, \cdot_A)$ i $(B, +_B, \cdot_B)$ to pierścienie oraz

$$
\forall_{a \in A, b \in B} f(a +_A b) = f(a) +_B f(b) \wedge f(a \cdot_B b) = f(a) \cdot_B f(b)
$$

\subsection{Jądro homomorfizmu}

$$
ker f = \{a \in A: f(a) = O_B\} 
$$

\subsection{Obraz homomorfizmu}

$$
im f = \{b \in B: \exists a \in A: f(a) = b\}
$$

\section{Permutacje}

$$
\pi =
\begin{pmatrix}
  1   & 2   & 3   & \cdots & n-1   \\
  a_1 & a_2 & a_3 & \cdots &  a_n  
\end{pmatrix}
$$

$$
a_n = \pi(n)
$$

\subsection{Rozkład na cykle}

$$
\pi = 
\begin{pmatrix}
    a_1 & a_2 & a_3 & \cdots & a_n   \\
    a_2 & a_3 & a_4 & \cdots & a_1  
\end{pmatrix}
= (a_1, a_2, a_3, \dots, a_n)
$$

$$
\pi' = 
\begin{pmatrix}
    a_1 & a_2 & b_1 & b_2   \\
    a_2 & a_1 & b_2 & a_1  
\end{pmatrix}
= (a_1, a_2) \cdot (b_1, b_2)
$$

\subsection{Iloczyn transpozycji}

$$
(a_1, a_2, a_3, \dots, a_k) = (a_1, a_k) \cdot (a_1, a_{k-1}) \cdot \dots \cdot (a_1, a_3) \cdot (a_1, a_2)
$$

\subsection{Postać macierzowa}

$$
\pi = 
\begin{pmatrix}
    1 & 2 & 3 & 4\\
    2 & 2 & 4 & 3
\end{pmatrix}
= 
\begin{vmatrix}
    0 & 0 & 0 & 0\\
    1 & 1 & 0 & 0\\
    0 & 0 & 0 & 1\\
    0 & 0 & 1 & 0
\end{vmatrix}
$$

\subsection{Znak permutacji}

Ilość czynników w iloczynie transpozycji określa parzystość permutacji.

$$(-1)^n$$ gdzie $n$ to ilość transpozycji

\section{Macierze}

\subsection{Macierz jednostkowa}

$$
I =
\begin{vmatrix}
    1 & 0 & 0 & \cdots & 0 \\
    0 & 1 & 0 & \cdots & 0 \\
    0 & 0 & 1 & \cdots & 0 \\
    \vdots & \vdots & \vdots & \ddots & \vdots \\
    0 & 0 & 0 & \cdots & 1
\end{vmatrix}
$$

\subsection{Macierz odwrotna}

Macierz odwrotna do $A$ to taka macierz $B$, że $A \cdot B = B \cdot A = I$

\subsection{Macierz transponowana}

$$
A^T =
\begin{vmatrix}
    a_{11} & a_{21} & a_{31} & \cdots & a_{m1} \\
    a_{12} & a_{22} & a_{32} & \cdots & a_{m2} \\
    a_{13} & a_{23} & a_{33} & \cdots & a_{m3} \\
    \vdots & \vdots & \vdots & \ddots & \vdots \\
    a_{1n} & a_{2n} & a_{3n} & \cdots & a_{mn}
\end{vmatrix}
$$

\subsection{Wzory Cramera}

\subsection{Ograniczenie macierzy}

$$
A_{ij} = \text{macierz bez kolumny } i \text{ oraz wierszu } j
$$

\subsection{Wyznacznik Macierzy}

$$
\det A = a_{11} \cdot A_{11} + a_{12} \cdot A_{12} + a_{13} \cdot A_{13} + \dots + a_{1n} \cdot A_{1n}
$$

Dla macierzy $2 \times 2$

$$
$$

Dla macierzy $3 \times 3$

$$
$$

\subsubsection{Tw. Laplace'a}

$$
\det A = a_{i1}A_{i1} + a_{i2}A_{i2} + \dots + a_{in}A_{in} \text{ dla każdego } 1 \le i \le n
$$

\subsubsection{Własności}

\begin{itemize}
    \item Jeżeli macierz kwadratowa $A$ ma wiersz lub kolumnę złożoną z samych zer to $\det A$ = 0
    \item Jeżeli macierz kwadratowa ma kolumnę lub wiersz pomnożoną przez skalar to wyznacznik też jest wielokrotnością skalara
    \item Jeżeli dwie macierze $A$ i $B$ kwadratowe różnią się od innej macierzy $C$ tylko tą samą kolumną lub wierszem, który $C$ jest sumą odpowiednich w $A$ i $B$ to $\det C = \det A + \det B$
    \item Zamiana miejscami dwóch kolumn lub wierszy spowoduje zamienienie się znaku wyznacznika na przeciwny
    \item Jeżeli jedna kolumna lub wiersz jest wielokrotnością innego wiersza lub kolumny to wyznacznik jest równy 0
    \item Dodawanie wierszy i kolumn nie zmienia wyznacznika
    \item Wyznacznik macierzy górnotrójkątnej lub dolnotrójkątnej jest równy iloczynowi elementów na przekątnej
\end{itemize}

\subsubsection{Tw. Cauche'go}

$$
\det AB = \det A \cdot \det B
$$

\subsection{Dopełnienie algebraiczne macierzy}

$$
A_{ij} = (-1)^{i + j} \det M_{ij} 
$$

\end{document}