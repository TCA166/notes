\documentclass{../notatki}

\title{Algebra liniowa}

\begin{document}

\tableofcontents

\section{Uporządkowana para liczb}

$$
(a, b) = \{ \{a\}, \{a, b\} \}
$$

\section{Grupa}

Grupa to uporządkowana para $G(A, \circ)$, gdzie $A$ to zbiór, a
$\circ$ to działanie spełniające następujące warunki:

\begin{itemize}
  \item Zachodzi łączność działania $\forall a, b, c \in A: a \circ
    (b \circ c) = (a \circ b) \circ c$
  \item Istnieje element neutralny $e \in A: \forall a \in A: a \circ
    e = e \circ a = a$
  \item Dla każdego elementu istnieje element odwrotny $\forall a \in
    A: \exists a^{-1} \in A: a \circ a^{-1} = a^{-1} \circ a = e$
\end{itemize}

\subsection{Grupa abelowa}

Grupa abelowa, to specjalny rodzaj grupy w którym spełniony jest
dodatkowy warunek:

\begin{itemize}
  \item Grupa jest przemienna $\forall a, b \in A: a \circ b = b \circ a$
\end{itemize}

\subsection{Przykłady grup}

$$
G(\mathbb{Z}, +), G(\mathbb{Q}, +), G(\mathbb{R}, +), G(\mathbb{C}, +)
$$

\section{Pierścień}

Pierścień to uporządkowana trójka $R(A, +, \cdot)$, gdzie $A$ to
zbiór, a $+$ i $\cdot$ to działania spełniające następujące warunki:

\begin{itemize}
  \item $(A, +)$ jest grupą abelową
  \item $+$ i $\cdot$ są są wewnętrzne dla $A$
  \item Dla każdego $a, b, c \in A$ zachodzi rozdzielność mnożenia
    względem dodawania: $a \cdot (b + c) = a \cdot b + a \cdot c$
    oraz $(a + b) \cdot c = a \cdot c + b \cdot c$
  \item Istnieje element neutralny mnożenia $1 \in A: \forall a \in
    A: a \cdot 1 = 1 \cdot a = a$
\end{itemize}

\subsection{Pierścień z jedynką}

Pierścień z jedynką to pierścień, w którym istnieje element neutralny
mnożenia oraz $A \ne \emptyset$

\subsection{Pierścień przemienny}

Pierścień przemienny to pierścień, w którym mnożenie jest przemienna

\section{Ciało}

Ciało $C(K, +, \cdot)$ to pierścień przemienny z jedynką, oraz $(K
\setminus \{0\}, \cdot)$ jest grupą

\section{Homomorfizmy}

Homomorfizmy to odwzorowania $f: A \rightarrow B$, jeśli $A$ i $B$
spełniają dodatkowe warunki..

\subsection{Homomorfizmy grupy}

Jeśli $(A, +_A)$ i $(B, +_B)$ to grupy oraz

$$
\forall_{a \in A, b \in B} f(a +_A b) = f(a) +_B f(b)
$$

\subsection{Homomorfizmy pierścieni}

Jeśli $(A, +_A, \cdot_A)$ i $(B, +_B, \cdot_B)$ to pierścienie oraz

$$
\forall_{a \in A, b \in B} f(a +_A b) = f(a) +_B f(b) \wedge f(a
\cdot_B b) = f(a) \cdot_B f(b)
$$

\subsection{Jądro homomorfizmu}

$$
ker f = \{a \in A: f(a) = O_B\}
$$

\subsection{Obraz homomorfizmu}

$$
im f = \{b \in B: \exists a \in A: f(a) = b\}
$$

\section{Permutacje}

$$
\pi =
\begin{pmatrix}
  1   & 2   & 3   & \cdots & n-1   \\
  a_1 & a_2 & a_3 & \cdots &  a_n
\end{pmatrix}
$$

$$
a_n = \pi(n)
$$

\subsection{Rozkład na cykle}

$$
\pi =
\begin{pmatrix}
  a_1 & a_2 & a_3 & \cdots & a_n   \\
  a_2 & a_3 & a_4 & \cdots & a_1
\end{pmatrix}
= (a_1, a_2, a_3, \dots, a_n)
$$

$$
\pi' =
\begin{pmatrix}
  a_1 & a_2 & b_1 & b_2   \\
  a_2 & a_1 & b_2 & a_1
\end{pmatrix}
= (a_1, a_2) \cdot (b_1, b_2)
$$

\subsection{Iloczyn transpozycji}

$$
(a_1, a_2, a_3, \dots, a_k) = (a_1, a_k) \cdot (a_1, a_{k-1}) \cdot
\dots \cdot (a_1, a_3) \cdot (a_1, a_2)
$$

\subsection{Postać macierzowa}

$$
\pi =
\begin{pmatrix}
  1 & 2 & 3 & 4\\
  2 & 2 & 4 & 3
\end{pmatrix}
=
\begin{bmatrix}
  0 & 0 & 0 & 0\\
  1 & 1 & 0 & 0\\
  0 & 0 & 0 & 1\\
  0 & 0 & 1 & 0
\end{bmatrix}
$$

\subsection{Znak permutacji}

Ilość czynników w iloczynie transpozycji określa parzystość permutacji.

$$(-1)^n$$ gdzie $n$ to ilość transpozycji

\section{Macierze}

\subsection{Macierz jednostkowa}

$$
I =
\begin{bmatrix}
  1 & 0 & 0 & \cdots & 0 \\
  0 & 1 & 0 & \cdots & 0 \\
  0 & 0 & 1 & \cdots & 0 \\
  \vdots & \vdots & \vdots & \ddots & \vdots \\
  0 & 0 & 0 & \cdots & 1
\end{bmatrix}
$$

\subsection{Macierz odwrotna}

Macierz odwrotna do $A$ to taka macierz $B$, że $A \cdot B = B \cdot A = I$

\subsection{Macierz transponowana}

$$
A^T =
\begin{bmatrix}
  a_{11} & a_{21} & a_{31} & \cdots & a_{m1} \\
  a_{12} & a_{22} & a_{32} & \cdots & a_{m2} \\
  a_{13} & a_{23} & a_{33} & \cdots & a_{m3} \\
  \vdots & \vdots & \vdots & \ddots & \vdots \\
  a_{1n} & a_{2n} & a_{3n} & \cdots & a_{mn}
\end{bmatrix}
$$

\subsection{Minory macierzy}

$$
A_{ij} = \text{macierz bez kolumny } i \text{ oraz wierszu } j
$$

\subsection{Wyznacznik Macierzy}

$$
\det A = a_{11} \cdot A_{11} + a_{12} \cdot A_{12} + a_{13} \cdot
A_{13} + \dots + a_{1n} \cdot A_{1n}
$$

Dla macierzy $2 \times 2$

$$
\det A = a_{11} \cdot a_{22} - a_{12} \cdot a_{21}
$$

Dla macierzy $n \times n$

$$
\det A = \sum_{i=1}^{n} (-1)^{i+j} \cdot a_{ij} \cdot A_{ij}
$$

\subsubsection{Tw. Laplace'a}

$$
\det A = a_{i1}A_{i1} + a_{i2}A_{i2} + \dots + a_{in}A_{in} \text{
dla każdego } 1 \le i \le n
$$

\subsubsection{Własności}

\begin{itemize}
  \item Jeżeli macierz kwadratowa $A$ ma wiersz lub kolumnę złożoną z
    samych zer to $\det A$ = 0
  \item Jeżeli macierz kwadratowa ma kolumnę lub wiersz pomnożoną
    przez skalar to wyznacznik też jest wielokrotnością skalara
  \item Jeżeli dwie macierze $A$ i $B$ kwadratowe różnią się od innej
    macierzy $C$ tylko tą samą kolumną lub wierszem, który $C$ jest
    sumą odpowiednich w $A$ i $B$ to $\det C = \det A + \det B$
  \item Zamiana miejscami dwóch kolumn lub wierszy spowoduje
    zamienienie się znaku wyznacznika na przeciwny
  \item Jeżeli jedna kolumna lub wiersz jest wielokrotnością innego
    wiersza lub kolumny to wyznacznik jest równy 0
  \item Dodawanie wierszy i kolumn nie zmienia wyznacznika
  \item Wyznacznik macierzy górnotrójkątnej lub dolnotrójkątnej jest
    równy iloczynowi elementów na przekątnej
\end{itemize}

\subsubsection{Tw. Cauche'go}

$$
\det AB = \det A \cdot \det B
$$

\subsection{Wzory Cramera}

Dla zestawu równań liniowych

$$
\begin{cases}
  a_{11}x_1 + a_{12}x_2 + \dots + a_{1n}x_n = b_1 \\
  a_{21}x_1 + a_{22}x_2 + \dots + a_{2n}x_n = b_2 \\
  \vdots \\
  a_{n1}x_1 + a_{n2}x_2 + \dots + a_{nn}x_n = b_n
\end{cases}
$$
Możemy przedstawić czynniki jako macierz
$$
W = \det
\begin{bmatrix}
  a_{11} & a_{12} & \cdots & a_{1n} \\
  a_{21} & a_{22} & \cdots & a_{2n} \\
  \vdots & \vdots & \ddots & \vdots \\
  a_{n1} & a_{n2} & \cdots & a_{nn}
\end{bmatrix}
$$
I zastępując kolumnę $i$ kolumną wyrazów wolnych otrzymujemy
$$
W_i = \det
\begin{bmatrix}
  a_{11} & a_{12} & \cdots & a_{1i-1} & b_1 & a_{1i+1} & \cdots & a_{1n} \\
  a_{21} & a_{22} & \cdots & a_{2i-1} & b_2 & a_{2i+1} & \cdots & a_{2n} \\
  \vdots & \vdots & \ddots & \vdots & \vdots & \vdots & \ddots & \vdots \\
  a_{n1} & a_{n2} & \cdots & a_{ni-1} & b_n & a_{ni+1} & \cdots & a_{nn}
\end{bmatrix}
$$

$$
x_i = \frac{W_i}{W}
$$

\subsection{Dopełnienie algebraiczne macierzy}

$$
A_{ij} = (-1)^{i + j} \det M_{ij}
$$

\section{Przestrzeń liniowa}

Przestrzeń liniowa $V$ nad ciałem $K$ to zbiór $V$ oraz działania $+$
i $\cdot$ spełniające następujące warunki:

\begin{itemize}
  \item $(V, +, \theta)$ jest grupą abelową z elementem neutralnym $\theta$
  \item Dla każdego $\alpha, \beta \in K$ i $v \in V$ zachodzi
    $\alpha \cdot (\beta \cdot v) = (\alpha \cdot \beta) \cdot v$
  \item Dla każdego $\alpha \in K$ i $v, w \in V$ zachodzi $\alpha
    \cdot (v + w) = \alpha \cdot v + \alpha \cdot w$
  \item $1 \cdot v = v$
\end{itemize}

\section{Wektory}

Wektor to element przestrzeni liniowej $V$

\subsection{Układ wektorów}

Układ wektorów przestrzeni liniowej $V$ o wskaźnikach ze zbioru T to
funkcja $v: T \rightarrow V$. Wartość funkcji v w elemencie t oznaczamy $v_t$.

\subsection{Kombinacja liniowa}

Kombinacja liniowa wektorów $v_1, v_2, \dots, v_n$ to wektor postaci

$$
\alpha_1 \cdot v_1 + \alpha_2 \cdot v_2 + \dots + \alpha_n \cdot v_n
$$

Zbiór wszystkich kombinacji liniowych nazywamy powłoką liniową układu

\subsection{Rozpinanie}

Wektory z układu wektorów rozpinają przestrzeń jeżeli każdy z
wektorów należy do przestrzeni oraz jest kombinacją liniową wektorów
układu, czyli można go wyrazić za pomocą innych wektorów.

Wektory są liniowo niezależne gdy ich kombinacja liniowa (L) jest
równa $\theta_v$(wektor zerowy) lub matryca tych wektorów ma $\det
\ne 0$ lub rząd matrycy wektorów = ilości wektorów

\subsection{Własności wektorów}

\begin{itemize}
  \item Jeżeli wektory są liniowo zależne to jeden z nich jest
    kombinacją liniową pozostałych
  \item Jeżeli $\theta_v$ jest kombinacją liniową wektorów to są liniowo zależne
  \item Kolumny macierzy $A$ są wektorami liniowo zależnymi jeżeli
    $AX = \theta$ ma rozwiązanie niezerowe ze względu na $X$
  \item Jeżeli kolumny macierzy $A$ to wektory liniowo niezależne to
    $AX = \theta$ ma jedno rozwiązanie i $\det A \ne 0$
\end{itemize}

\section{Baza}

Niezależny układ rozpinający przestrzeń V nazywamy bazą przestrzeni
V. Jeżeli układ tworzący bazę jest skończony to mówimy że V ma
bazę skończoną. Liczba elementów bazy $S$ przestrzeni $V$ nazywamy $dim V$ czyli
wymiarem przestrzeni.

\subsection{Macierz przejścia}

Dla dwóch baz $B_1= (v_1, \dots, v_n), B_2 = (w_1, \dots, w_n)$ przestrzeni
liniowej $R_n$ oraz zależności $v_n = a_{1n}w1 + a_{2n}w_2 + \dots + a_{nn}w_n$
to macierz
$$
A =
\begin{bmatrix}
  a_{11} & a_{12} & \cdots & a_{1n} \\
  a_{21} & a_{22} & \cdots & a_{2n} \\
  \vdots & \vdots & \ddots & \vdots\\
  a_{n1} & a_{n2} & \cdots & a_{nn}
\end{bmatrix}
$$
jest macierzą przejścia między $B_1$ i $B_2$

\section{Wrońskian}

Macierz zawierająca pochodne n funkcji do m-1 włącznie stopnia.

$$
W_{11} = f_1(x), W_{12} = f_2(x), \dots, W_{1n} = f_n(x)
$$

$$
W_{21} = f_1'(x), W_{22} = f_2'(x), \dots, W_{2n} = f_n'(x)
$$

Jeśli funkcje są liniowo zależne w przedziale $(a,b)$ to ich
wrońskian jest tożsamościowo równy zeru.
Jeżeli $x_0 \in (a, b)$ i $W(x_0) \ne 0$ to funkcje we wrońskianie są
liniowo niezależne

\section{Przekształcenie liniowe}

Mówimy, że funkcja $T: V \rightarrow W$ jest przekształceniem liniowym, jeżeli:

$$
\forall_{v_1, v_2 \in V} T(v_1 + v_2) = T(v_1) + T(v_2) \land
T(\alpha \cdot v) = \alpha T(v)
$$

Jeżeli $V = W$, to przekształcenie liniowe nazywamy operatorem
liniowym przestrzeni.

\subsection{Macierz przekształcenia liniowego}

Dla dwóch baz $B_v i B_w$ oraz $T(v_n) = a_{1n}w_1 + a_{2n}w_2 +
\dots + a_{mn}w_m$.
To macierz wymiaru $m \times n$ to macierz przekształcenia $T$ w bazach.

\section{Wektory własne}

Niezerowy wektor $v$ nazywamy wektorem własnym macierzy $A$ jeżeli istnieje
skalar $\lambda$ taki, że $Av = \lambda v$. W takiej konfiguracji $\lambda$
nazywamy wartością własną macierzy $A$ odpowiadającą wektorowi $v$.

\subsection{Macierz charakterystyczna}

Macierz $A - tI_n$ o współczynnikach w pierścieniu wielomianów $K[t]$ nazywamy
macierzą charakterystyczną macierzy $A$. Równanie $\det(A - tI_n) = 0$ nazywamy
równaniem charakterystycznym macierzy. $f_A(t) = \det(A -
tI_n)=(-1)^n\det(A - tI_n) \in K[t]$ nazywamy wielomianem charakterystycznym
macierzy. $f_A(t) = (t - \lambda)^kg(t)$ nazywamy krotnością
algebraiczną wartością
własnej. Przestrzeń własna to zbiór wektorów własnych odpowiadającym wartościom
własnym.

Ślad macierzy to $tr A = a_{11} + a_{22} + \dots$. Krotność geometryczna
wartości własnej to rząd macierzy powstałej z wektora przestrzeni własnej danej
wartości własnej, lub wymiar przestrzeni własnej.

\end{document}