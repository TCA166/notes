\documentclass{../notatki}

\title{Analiza Matematyczna z Zastosowaniami 2}

\begin{document}

\section{Szereg Funkcyjny}

Dla ciągu funkcji rzeczywistych $f_n$, gdzie $D \subset \mathbb{R}$ i
$f_n: D \to \mathbb{R}$; to:
$$
S = \sum_{n=1}^{\infty} f_n(x) = \lim_{N \to \infty} \sum_{n=1}^{N} f_n(x)
$$
jest szeregiem funkcyjnym, na podstawie ciągu funkcyjnego $f_n(x)$.

\subsection{Zbieżność}

Dla ciągu funkcyjnego $f_n$ określonym na zbiorze $A$, mówimy, że jest
punktowo zbieżny do funkcji $f$ określonej w $A$ gdy $\lim_{n \to
\infty} f_n(x) = f(x)$.

Ciąg funkcyjny może na podzbiorze $E \subseteq D$ może być:
\begin{itemize}
  \item zbieżny punktowo, czyli zbieżny dla każdego $x \in E$
  \item zbieżny punktowo bezwględnie, to znaczy zbieżny punktowo dla $|f_n(x)|$
  \item zbieżny jednostajnie, to znaczy $a_n = \sup_{x \in E} f_n(x)$
    jest zbieżny
    $$
    \forall_{\epsilon > 0} \exists_{n_0} \forall_{n \ge n_0}
    \forall_{x \in E} |f(x) - f_n(x)| < \epsilon
    $$
    Geometrycznie: w pasie o brzegach $y = f(x) \pm \epsilon$ leżą wszystkie
    krzywe $y = f_n(x)$.
  \item zbieżny jednostajnie bezwględnie, to znaczy $a_n = \sup_{x
    \in E} |f_n(x)|$ jest zbieżny
\end{itemize}

Przykładem ciągu funkcyjnego, który jest zbieżny to $f_n(x) =
\frac{\sin(nx)}{n}$. Z kolei, na przykład; ciąg $f_n = x^n$ w przedziale
$0 \le x < 1$ nie jest zbieżny jednostajnie dp (dowolnej) funkcji f, bo w
pobliżu punktu $x - 1$ krzywe $y - x^n$ nie leżą dowolnie blisko prostej
$y - 0$.

\subsection{Kryterium zbieżności jednostajnej}

Granicą ciągu funkcyjnego jest jakaś funkcja. Ciąg $f_n$ jest ciągiem funkcji
określonych na niepustym zbiorze $A$ i o wartościach rzeczywistych lub
zespolonych. Ciąg $f_n$ jest jednostajnie zbieżny w $A$ do $f$ wtedy i tylko
wtedy gdy $\sup_{x \in A} |f_n(x) - f(x)| \to 0: n \to \infty$.

\subsection{Kryterium Cauchy'ego}

Ciąg funkcji $f_n$ jest jednostajnie zbieżny w zbiorze $A$ wtedy i tylko wtedy
gdy dla każdego $\epsilon > 0$ istnieje wskaźnik $N(\epsilon)$, taki że dla
$x \in A$
$$
|f_n(x) - f_m(x)| \le \epsilon : n,m \ge N(\epsilon)
$$

\subsection{Zbieżność niemal jednostajna}

Szereg funkcyjny na zbiorze $E$ jest zbieżny niemal jednostajnie, jeśli dla
dowolnego właściwego przedziału domkniętego $I = <a, b>$, $I \subset E$ jest
zbieżny jednostajnie na $I$.

\subsection{Twierdzenie Arzeli-Ascoli'ego}

Jeżeli $f_n$ jest ciągiem funkcji rzeczywistych określonych na przedziale
zwartym, który jest wspólnie ograniczony i jednakowo ciągły, to zawiera
on podciąg zbieżny jednostajnie.

% W praktyce twierdzenie te może być stosowane do dowodu poprawności
% algorytmów kompresji i odtwarzania obrazów, zapewniając
% jednocześnie, że proces ten jest numerycznie stabilny i zachowuje
% jakość danych wizualnych.

\subsection{Ciągłość granicy}

Jeżeli ciąg $f_n(x)$ jest jednostajnie zbieżny w $A$ i funkcje $f_n(x)$ są
funkcjami ciągłymi w punkcie $x_0 \in A$, to funkcja graniczna $f(x)
= \lim_{n \to \infty} f_n(x)$ jest ciągła w punkcie $x_0$.

\subsection{Kryteria zbieżności jednostajnej szeregów}

\subsubsection{Kryterium porównawcze Weierstrassa}

Jeśli szereg liczbowy utworzony z ciągu $a_n$ o wyrazach nieujemnych jest
zbieżny oraz dla każðego $x \in E$ zachodzi nierówność $|f_n(x)| < a_n$, to
szereg funkcyjny $\sum_{n=1}^\infty f_n$ jest bezwględnie jednostajnie
zbieżny na $E$.

\subsubsection{Kryterium Dirichleta}

Jeśli ciąg sum częściowych $S_n$ utworzony z ciągu funkcyjnego $f_n$ jest
wspólnie ograniczony ($S_n(x) < M : x \in E, n \in \mathbb{N}$), oraz jeśli
ciąg funkcyjny zbiega do zera monotonicznie i jednostajnie to szereg
$\sum_{n=1}^\infty f_n g_n$

\subsubsection{Kryterium Abela}

Niech $a_n(x)$ oraz $b_n(x)$ będą ciągami funkcyjnymi określonymi w zbiorze
$A$. Jeśli ciąg $a_n$ jest monotoniczny dla każdego $x$, oraz jest
ciągiem wspólnie ograniczonym oraz szereg $\sum_{n=1}^\infty b_n(x)$
jest jednostajnie zbieżny w $A$. Wtedy $\sum_{n=1}^{\infty} a_n b_n$ jest
jednoznacznie zbieżny w $A$.

\end{document}
