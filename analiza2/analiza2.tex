\documentclass{../notatki}

\title{Analiza Matematyczna z Zastosowaniami 2}

\begin{document}

\section{Szereg Funkcyjny}

Dla ciągu funkcji rzeczywistych $f_n$, gdzie $D \subset \mathbb{R}$ i
$f_n: D \to \mathbb{R}$; to:
$$
S = \sum_{n=1}^{\infty} f_n(x) = \lim_{N \to \infty} \sum_{n=1}^{N} f_n(x)
$$
jest szeregiem funkcyjnym, na podstawie ciągu funkcyjnego $f_n(x)$.

Ciąg funkcyjny może na podzbiorze $E \subseteq D$ może być:
\begin{itemize}
  \item zbieżny punktowo, czyli zbieżny dla każdego $x \in E$
  \item zbieżny punktowo bezwględnie, to znaczy zbieżny punktowo dla $|f_n(x)|$
  \item zbieżny jednostajnie, to znaczy $a_n = \sup_{x \in E} f_n(x)$
    jest zbieżny
  \item zbieżny jednostajnie bezwględnie, to znaczy $a_n = \sup_{x
    \in E} |f_n(x)|$ jest zbieżny
\end{itemize}

Przykładem ciągu funkcyjnego, który jest zbieżny to $f_n(x) =
\frac{\sin(nx)}{n}$

\end{document}
