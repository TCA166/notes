\documentclass{../notatki}

\title{Analiza Matematyczna z Zastosowaniami 2}

\begin{document}

\section{Ciąg Funkcyjny}

Ciąg funkcyjny, to ciąg funkcji $f_n(x)$ określonych na pewnym zbiorze $A$.
Przykład:
$$
f_n(x) = \frac{x}{n}
$$

\subsection{Zbieżność}

Ciąg funkcyjny może na podzbiorze $E \subseteq D$ może być:
\begin{itemize}
  \item zbieżny punktowo, czyli zbieżny dla każdego $x \in E$
    $$
    \forall_{x \in E} \lim_{n \to \infty} f_n(x) = f(x)
    $$
    $$
    \forall_{x \in E} f(x) = \lim_{n \to \infty} f_n(x)
    $$
    $$
    \forall_{x \in E} \forall_{\epsilon > 0} \exists_{N} \forall_{n
    \ge N} |f_n(x) - f(x)| < \epsilon
    $$
    Lub; dla każdego $x \in E$, dla każdego marginesu, od pewnego
    punktu, odległość między $f_n$ i $f$ w punkcie jest mniejsza niż $\epsilon$.
    Ponieważ w definicji $\epsilon$ zależy od $x$, to $f$ może nie być ciągła.
  \item zbieżny punktowo bezwględnie, to znaczy zbieżny punktowo dla $|f_n(x)|$
  \item zbieżny jednostajnie, to znaczy $a_n = \sup_{x \in E} f_n(x)$
    jest zbieżny
    $$
    \lim_{n \to \infty} \sup_{x \in E} |f_n(x) - f(x)| = 0
    $$
    $$
    \forall_{\epsilon > 0} \exists_{N} \forall_{n \ge N} \forall_{x
    \in E} |f(x) - f_n(x)| < \epsilon
    $$
    Geometrycznie: w pasie o brzegach $y = f(x) \pm \epsilon$ leżą wszystkie
    krzywe $y = f_n(x)$. Zbieżność jednostajna implikuje punktową, oraz
    wymaga ciągłość $f$. Granica ciągu jednostajnie zbieżnego jest ciągła.
  \item zbieżny jednostajnie bezwględnie, to znaczy $a_n = \sup_{x
    \in E} |f_n(x)|$ jest zbieżny
\end{itemize}

Przykładem ciągu funkcyjnego, który jest zbieżny to $f_n(x) =
\frac{\sin(nx)}{n}$. Z kolei, na przykład; ciąg $f_n = x^n$ w przedziale
$0 \le x < 1$ nie jest zbieżny jednostajnie dp (dowolnej) funkcji f, bo w
pobliżu punktu $x - 1$ krzywe $y - x^n$ nie leżą dowolnie blisko prostej
$y - 0$.

\subsection{Szereg Funkcyjny}

Niech $f_n(x)$ będzie ciągiem funkcji określonych na zbiorze $E$. Jeżeli dla
każdego $x \in E$ szereg liczbowy $\sum_{k=0}^\infty f_n(x)$ jest zbieżny, to
funkcja $f(x) = \sum_{k=0}^\infty f_n(x)$ jest sumą szeregu
funkcyjnego. Zbieżność ciągu $a_n$ jest warunkiem koniecznym, ale nie
wystarczającym dla zbieżności szeregu, lub inaczej, zbieżność szeregu
implikuje zbieżność ciągu $a_n$. W sprawdzaniu zbieżności szeregu, bardzo
przydaje się wzór na sumę szeregu geometrycznego, gdzie $|q| < 1$:
$\frac{a_1}{1-q}$.

Jeżeli ciąg sum częściowych $S_n(x) = \sum_{k=0}^n f_k(x)$ jest zbieżny
jednostajnie, to mówimy, że szereg tworzony przez ten ciąg jest zbieżny
jednostajnie. Suma szeregu jednostajnie zbieżnego jest ciągła.

\subsection{Twierdzenie Arzeli-Ascoli'ego}

Jeżeli $f_n$ jest ciągiem funkcji rzeczywistych określonych na przedziale
zwartym, który jest wspólnie ograniczony i jednakowo ciągły, to zawiera
on podciąg zbieżny jednostajnie.

% W praktyce twierdzenie te może być stosowane do dowodu poprawności
% algorytmów kompresji i odtwarzania obrazów, zapewniając
% jednocześnie, że proces ten jest numerycznie stabilny i zachowuje
% jakość danych wizualnych.

\subsection{Ciągłość granicy}

Jeżeli ciąg $f_n(x)$ jest jednostajnie zbieżny w $A$ i funkcje $f_n(x)$ są
funkcjami ciągłymi w punkcie $x_0 \in A$, to funkcja graniczna $f(x)
= \lim_{n \to \infty} f_n(x)$ jest ciągła w punkcie $x_0$.

\subsection{Ciągłość a całka}

Jeśli $f_n$ jest ciągiem funkcji ciągłych, to:
\begin{itemize}
  \item Dla zbieżnego jednostajnie ciągłego ciągu $f_n(x)$:
    $$
    \int_E f(x) dx = \lim_{n \to \infty} \int_E f_n(x) dx
    $$
  \item Dla zbieżnego jednostajnie ciągłego $\sum_{n=1}^\infty f_n(x)$:
    $$
    \int_E \sum_{n=1}^\infty f_n(x) dx = \lim_{n \to \infty} \int_E
    \sum_{k=1}^n f_k(x) dx
    $$
\end{itemize}

\subsection{Kryteria zbieżności}

\subsubsection{Kryterium porównawcze Weierstrassa}

Jeśli szereg liczbowy utworzony z ciągu $a_n$ o wyrazach nieujemnych jest
zbieżny oraz dla każðego $x \in E$ zachodzi nierówność $|f_n(x)| < a_n$, to
szereg funkcyjny $\sum_{n=1}^\infty f_n$ jest bezwględnie jednostajnie
zbieżny na $E$.

\subsubsection{Kryterium Dirichleta}

Jeżeli $b_n \to 0$ i jeśli sumy częściowe $\sum_{n=1}^\infty a_n$ są
ograniczone,
to $\sum_{n=1}^\infty a_n b_n$ jest zbieżny.

\subsubsection{Kryterium Abela}

Niech $a_n(x)$ oraz $b_n(x)$ będą ciągami funkcyjnymi określonymi w zbiorze
$A$. Jeśli ciąg $a_n$ jest monotoniczny dla każdego $x$, oraz jest
ciągiem wspólnie ograniczonym oraz szereg $\sum_{n=1}^\infty b_n(x)$
jest jednostajnie zbieżny w $A$. Wtedy $\sum_{n=1}^{\infty} a_n b_n$ jest
jednoznacznie zbieżny w $A$.

\section{Szeregi potęgowe}

$$
S(x) = \sum_{n=0}^\infty a_n (x - x_0)^n
$$

\subsection{Twierdzenie Cauchy'ego-Hadamarda}

Dla szeregu potęgowego $\sum_{n=0}^\infty a_n (z - z_0)^n$, niech
$\lambda = \limsup_{n \to \infty} \sqrt[n]{|a_n|}$. Wtedy promieniem
zbieżności szeregu potęgowego jest $R = \frac{1}{\lambda}$. Za wyjątkiem
$\lambda = 0$, gdzie $R = \infty$, oraz $\lambda = \infty$, gdzie $R = 0$.
Szereg jest zbieżny dla $|z - z_0| < R$ i rozbieżny dla $|z - z_0| > R$.
Zbieżność w punkcie $z = z_0$ zależy od ciągu $a_n$.

$$
\lambda = \lim_{n \to \infty} \frac{|a_{n+1}|}{|a_n|}
$$

\subsection{Wzór Eulera}

$$
e^{ix} = \cos x + i \sin x
$$
Możemy też ten wzór dalej rozwijać, więdząc, że:
$$
\cos x = \frac{e^{ix} + e^{-ix}}{2}
$$

$$
\sin x = \frac{e^{ix} - e^{-ix}}{2i}
$$

\subsection{Szeregi Taylora}

$$
\sin x = \sum_{n=0}^\infty \frac{(-1)^n}{(2n+1)!} x^{2n+1}
$$

$$
\cos x = \sum_{n=0}^\infty \frac{(-1)^n}{(2n)!} x^{2n}
$$

$$
e^x = \sum_{n=0}^\infty \frac{x^n}{n!}
$$

\section{Przestrzenie metryczne}

Jeżeli na zbiorze $X$ określono funkcję $d: X \times X \rightarrow
\mathbb{R}$, która spełnia następujące warunki:
\begin{enumerate}
  \item $d(x, x) = 0$
  \item $d(x, y) = d(y, x)$
  \item $d(x, y) \leq d(x, z) + d(z, y)$
\end{enumerate}
to $(X, d)$ nazywamy przestrzenią metryczną.

\subsection{Zbieżność w przestrzeni metrycznej}

Mówimy, że ciąg $x_n$ jest zbieżny do $x$ w przestrzeni metrycznej $(X, d)$,
jeśli $\lim_{n \to \infty} d(x_n, x) = 0$.

\subsection{Twierdzenie Banacha o punkcie stałym}

Mówimy, że funkcja $f: X \rightarrow X$ jest kontrakcją, jeśli
istnieje taka liczba $L \in (0, 1)$, że dla każdego $x, y \in X$ zachodzi
$$
d(f(x), f(y)) \leq L d(x, y)
$$

Jeśli $X$ to przestrzeń metryczna, a $f$ jest kontrakcją ze stałą $L
\in (0, 1)$, to $f$ ma punkt stały $x_0$ w $X$. $x_0$ jest granicą ciągu
$x_1 \in X$, $x_{n + 1} = f(x_n)$. Co więcej:
$$
d(x_n, x_0) \leq \frac{L^{n-1}}{1 - L} d(x_2, x_1)
$$

\section{Szeregi Fouriera}

Jeżeli szereg trygonometryczny
$$
\frac{a_0}{2} + \sum_{n=1}^\infty (a_n \cos nx + b_n \sin nx)
$$
jest zbieżny jednostajnie do funkcji $f$ na przedziale $[-\pi, \pi]$, to
$$
a_0 = \frac{1}{\pi} \int_{-\pi}^\pi f(x) dx
$$
$$
a_n = \frac{1}{\pi} \int_{-\pi}^\pi f(x) \cos nx \, dx
$$
$$
b_n = \frac{1}{\pi} \int_{-\pi}^\pi f(x) \sin nx \, dx
$$

\subsection{Funkcja kawałkami gładka}

Mówimy, że funkcja $f$ ma nieciągłość skokową w punkcie $x_0$, jeśli
$$
\lim_{x \to x_0^+} f(x) - \lim_{x \to x_0^-} f(x) \neq 0
$$
Mówimy, że funkcja $f$ jest kawałkami gładka, jeśli w dowolnym przedziale
$[a, b]$ jest ciągła poza skończoną liczbą punktów przedziału $(a, b)$,
jej punkty nieciągłości są punktami skokowymi, a pochodna funkcji
jest ciągła poza skończoną liczbą punktów przedziału $(a, b)$.

\subsection{Zbieżność szeregów Fouriera}

Jeżeli funkcja $2\pi$ okresowa $f$ jest kawałkami gładka na $\mathbb{R}$, to
jej szereg Fouriera jest zbieżny dla dowolnego $x \in \mathbb{R}$ do wartości:
$$
\frac{f(x-)+f(x+)}{2}
$$

\end{document}
