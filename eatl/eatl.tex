\documentclass{../notatki}

\title{Elementy algebry i teorii liczb}

\begin{document}

\tableofcontents

\noindent Kazali mi to zdawać choć algebrę miałem, jakbym nie miał ciekawszych
rzeczy do roboty i potrzebował tej powtórki. Fun times.

\section{Operacje}

Każdą funkcję która ma dwa argumenty i zwraca jeden wynik można nazwać
operacją. Teoretycznie zatem można konwencjonalne operatory traktować jako
funkcje. $+(1, 1)=2$

\subsection{Działanie wewnętrzne i zewnętrzne}

Działanie wewnętrzne w zbiorze $A$: $*: A \times A \rightarrow A$.
Działanie zewnętrzne w zbiorze $A$: $*: F \times A \rightarrow A$

\subsection{Własności operacji}

Rozróżniamy kilka własności, które mogą mieć operacje.

\begin{itemize}
  \item \textbf{Łączność} - $A * (B * C) = (A * B) * C$
  \item \textbf{Przemienność} - $A * B = B * A$
  \item \textbf{Rozdzielność} - $A * (B + C) = A * B + A * C$
  \item \textbf{Element neutralny} - $A * E = A$
  \item \textbf{Element odwrotny} - $A * A^{-1} = E$
\end{itemize}

\section{Grupa}

Grupa to zbiór $G$ z działaniem wewnętrznym $*$ jeśli:
\begin{itemize}
  \item $*$ jest łączne
  \item $*$ posiada element neutralny
  \item $*$ posiada element odwrotny
\end{itemize}
Dodatkowo jeśli $*$ jest przemienne to mamy grupę abelową.

\subsection{\texorpdfstring{Grupa $\mathbb{Z}_n$}{Grupa Zn}}

Specyficzna grupa, która jest zbiorem liczb całkowitych od $0$ do $n-1$ z
działaniem $+$ modulo $n$. Elementem przeciwnym dla $a$ jest $n - a$.

\subsection{\texorpdfstring{Grupa $\mathbb{Z}_n^\times$}{Grupa Znx}}

$$
\mathbb{Z}_n^\times = \{a \in \mathbb{Z}_n : NWD(a,n) = 1\}
$$
A działanie tej grupy to mnożenie modulo $n$. Element przeciwny oblicza się
algorytmem Euklidesa.

\section{Podgrupa}

Podgrupa to podzbiór grupy z odpowiednio dostosowanym działaniem. Na przykład
podgrupą $\mathbb{Z}_{12}$ jest $(\{0, 4, 8\}, +)$, ponieważ nie
ma pary elementów z
podzbioru, które po dodaniu dałyby coś spoza podzbioru.

\subsection{Generowanie}

Niech $(G, *)$ będzie grupą z elementem neutralnym $E$. Wtedy:
$$
\langle g \rangle = \{\stackrel{n}{\overbrace{g * g * \dots * g}} : n
\in \mathbb{N}\} \cup \{E\} \cup \{\stackrel{m}{\overbrace{g^{-1} * g^{-1} *
  \dots * g^{-1}}} : m
\in \mathbb{N}\}
$$
Jeśli $G = \langle g \rangle$ dla pewnego $g$ to $G$ jest grupą
cykliczną. Rzędem
$g$ jest $|\langle g \rangle|$

W $\mathbb{Z}_{12}$ podgrupą generowaną przez $4$ jest $\{0, 4, 8\}$, a
$rz(4) = |\langle 4 \rangle|$. Z kolei $\langle 1 \rangle = \mathbb{Z}_{12}$
zatem $\mathbb{12}$ jest grupą cykliczną. Jeżeli $p$ jest liczbą pierwszą
to $\mathbb{Z}_p^\times$ jest grupą cykliczną.

\subsection{Przystawanie}

Jeśli dwa elementy $a, b$ są przystające w Grupie $G$ to $a \equiv b$. Na
przykład $32 \equiv 4$ w $\mathbb{Z}_7$, ponieważ $32 \mod 7 = 4$.
Przystawanie ($\mod n$) implikuje:
\begin{itemize}
  \item że $a$ i $b$ przy dzieleniu przez $n$ mają tę samą resztę
  \item $n$ dzieli $a - b$
  \item $a = b + nk$ dla pewnego $k \in \mathbb{Z}$
\end{itemize}

\noindent
Ogólnym rozwiązaniem kongruencji $x \equiv a \mod n$ jest:
$$
x = a + nk \text{ dla pewnego } k \in \mathbb{Z}
$$
$$
(9x \equiv 6 \mod 15) \stackrel{\div 3}{\rightarrow} (3 \equiv 2 \mod 5)
\stackrel{3^{-1}}{\rightarrow} 1 \equiv 4 \mod 5
$$
Dla $x^y \mod z$, gdzie $t$ jest długością cyklu: $x^y \mod z = y \mod t$.
Na przykład: $522^{2024} \mod 65 = 2^{2024} \mod 65 = 2024 \mod 7$.

\section{Funkcja Eulera}

$$
\varphi(n) =
\begin{cases}
  1 : n = 1\\
  |\mathbb{Z}_n^\times| : n > 1\\
\end{cases}
$$

Jeśli $p$ jest liczbą pierwszą to $\varphi(p^k) = p^k - p^{k-1}$ oraz
$\varphi(p) = p - 1$. Jeśli $NWD(m, n) = 1$ to $\varphi(mn) =
\varphi(m)\varphi(n)$.

% z algebry

\section{Permutacje}

$$
\pi =
\begin{pmatrix}
  1   & 2   & 3   & \cdots & n-1   \\
  a_1 & a_2 & a_3 & \cdots &  a_n
\end{pmatrix}
$$

$$
a_n = \pi(n)
$$

\subsection{Rozkład na cykle}

$$
\pi =
\begin{pmatrix}
  a_1 & a_2 & a_3 & \cdots & a_n   \\
  a_2 & a_3 & a_4 & \cdots & a_1
\end{pmatrix}
= (a_1, a_2, a_3, \dots, a_n)
$$

$$
\pi' =
\begin{pmatrix}
  a_1 & a_2 & b_1 & b_2   \\
  a_2 & a_1 & b_2 & a_1
\end{pmatrix}
= (a_1, a_2) \cdot (b_1, b_2)
$$

\subsection{Iloczyn transpozycji}

$$
(a_1, a_2, a_3, \dots, a_k) = (a_1, a_k) \cdot (a_1, a_{k-1}) \cdot
\dots \cdot (a_1, a_3) \cdot (a_1, a_2)
$$

\subsection{Postać macierzowa}

$$
\pi =
\begin{pmatrix}
  1 & 2 & 3 & 4\\
  2 & 2 & 4 & 3
\end{pmatrix}
=
\begin{bmatrix}
  0 & 0 & 0 & 0\\
  1 & 1 & 0 & 0\\
  0 & 0 & 0 & 1\\
  0 & 0 & 1 & 0
\end{bmatrix}
$$

\subsection{Znak permutacji}

Ilość czynników w iloczynie transpozycji określa parzystość permutacji.

$$(-1)^n$$ gdzie $n$ to ilość transpozycji

\section{Pierścień}

Pierścień to uporządkowana trójka $R(A, +, \cdot)$, gdzie $A$ to
zbiór, a $+$ i $\cdot$ to działania spełniające następujące warunki:

\begin{itemize}
  \item $(A, +)$ jest grupą abelową
  \item $+$ i $\cdot$ są są wewnętrzne dla $A$
  \item Dla każdego $a, b, c \in A$ zachodzi rozdzielność mnożenia
    względem dodawania: $a \cdot (b + c) = a \cdot b + a \cdot c$
    oraz $(a + b) \cdot c = a \cdot c + b \cdot c$
  \item Istnieje element neutralny mnożenia $1 \in A: \forall a \in
    A: a \cdot 1 = 1 \cdot a = a$
\end{itemize}

\subsection{Pierścień z jedynką}

Pierścień z jedynką to pierścień, w którym istnieje element neutralny
mnożenia oraz $A \ne \emptyset$

\subsection{Pierścień przemienny}

Pierścień przemienny to pierścień, w którym mnożenie jest przemienne

\section{Ciało}

Ciało $\mathbb{C}(K, +, \cdot)$ to pierścień przemienny z jedynką, oraz $(K
\setminus \{0\}, \cdot)$ jest grupą.
% koniec algebry
Innymi słowy: jest to niepusty zbiór $K$ z działaniami $+$ i $\cdot$, które
są przemienne, łączne, posiadają elementy neutralne i odwrotne, oraz
istnieją takie pary $(a, b)$ dla których:
$$
a + b = 0 \text{ oraz } a \cdot b = 1
$$
Przykładami ciał są: $\mathbb{Q}$, $\mathbb{R}$, $\mathbb{C}$.

\section{Wielomiany}

Mówimy, że liczba $z$ jest pierwiastkiem $n$-tego stopnia liczby $w$ jeśli
$$
z^n = w
$$
Każdy wielomian $f \in \mathbb{C}[x]$ stopnia $n$ ma $n$
pierwiastków. Jeśli $f(x) = a_n x^n + a_{n-1} x^{n-1} + \dots + a_0$ to
$$
f(x) = a_n (x - z_1)(x - z_2) \dots (x - z_n)
$$

\subsection{Przykład ciała wielomianowego}

Zbiór $\{0, 1, x, x+1\}$ z dodawaniem i mnożeniem modulo $f(x) = x^2
+ x + 1 \in \mathbb{Z}_2[x]$ jest ciałem.

\begin{table*}[h]
  \centering
  \begin{tabular}{c|cccc}
    \hline
    $+$ & 0 & 1 & $x$ & $x + 1$ \\
    \hline
    0 & 0 & 1 & $x$ & $x + 1$ \\
    1 & 1 & 0 & $x + 1$ & $x$ \\
    $x$ & $x$ & $x + 1$ & 0 & 1 \\
    $x + 1$ & $x + 1$ & $x$ & 1 & 0 \\
    \hline
  \end{tabular}
  \caption{Dodawanie w wyżej zdefiniowanym ciele}
\end{table*}

Zbiór $\mathbb{Z}_n[x] / (f(x))$ jest ciałem wtedy i tylko wtedy gdy $f(x)$
jest nierozkładalny w $\mathbb{Z}_n[x]$.

\subsection{Rozkładalność a ciała}

Dlaczego zbiór $\{0, 1, x, x + 1\}$ z dodawaniem i mnożeniem modulo
$f(x) = x^2 + 1 \in \mathbb{Z}_2[x]$ nie jest ciałem? Ponieważ $x^2 +
1$ jest rozkładalny w $\mathbb{Z}_2[x]$.

Mówimy, że wielomian $f(x)$ jest rozkładalny w $\mathbb{Z}_p[x]$
jeśli gdy istnieją
wielomiany $g_1, g_2 \in \mathbb{Z}_p[x]$ stopnia co najmniej $1$ takie, że
$f(x) = g_1(x)g_2(x)$.

Dla każdego $n \in \mathbb{N}$ i każdej liczby pierwszej $p$ istnieje wielomian
stopnia $n$ w $\mathbb{Z}_p[x]$ który jest nierozkładalny.

\subsection{Ciała wielomianowe skończone}

Dla ciała $\mathbb{Z}_2[x]$, możemy wybrać dowolny nierozkładalny wielomian
stopnia $n$ i zdefiniować zbiór $\mathbb{Z}_2[x] / (f(x))$. Zbiór ten
będzie ciałem, ponieważ $f(x)$ jest nierozkładalny. Co więcej będzie miał
$2^n$ elementów, ponieważ $f(x)$ ma $n$ współczynników, a każdy z nich
może przyjąć $2$ wartości. Zbiór $\mathbb{Z}_2[x] / (f(x))$ nazywamy
\textbf{ciałem wielomianowym} i oznaczamy $\mathbb{F}_{2^n}$.

Na przykład $\mathbb{F}_8 = \mathbb{Z}_2[x]/(x^3 + x + 1)$, gdzie
$x^3 + x + 1$ jest nierozkładalny w $\mathbb{Z}_2[x]$. $\mathbb{F}_2$ to

\subsection{Wspólne miejsca zerowe wielomianów jednej zmiennej}

Mając wielomiany $f_1 \dots f_s \in \mathbb{F}[x]$ o współczynnikach
z ciała $\mathbb{F}$, chcemy znaleźć $V = \{x \in \mathbb{F}: f_{1
\dots s}(x) = 0 \}$.
$$
f(a) = 0 \leftrightarrow x - a | f(x)
$$
Aby znaleźć $V$ musimy obliczyć $NWD(f_1, \dots, f_s)$.

\subsection{Wielomiany wielu zmiennych}

$$
\mathbb{F}[x_1, \dots, x_n] = \text{zbiór wielomianów zmiennych }
x_1, \dots, x_n
$$

$$
f(x_1, \dots, x_n) \in \mathbb{F}[x_1, \dots, x_n] = \sum_{i_1, \dots,
i_n} a_{i_1, \dots, i_n} x_1^{i_1} \cdots x_n^{i_n}
$$

Konstrukcje typu $x_1^{i_1} \dots x_n^{i_n}$ można utożsamić z
wektorami $(i_1, \dots, i_n)$, a te z kolei uporządkować. Na przykład można użyć
porządku leksykograficznego gdzie $i \prec j \leftrightarrow
\text{pierwszy niezerowy współczynnik } j - a \text{ jest dodatni}$

Mając ustalony porządek, można zdefiniować dzielenie wielomianów
wielu zmiennych. Każdy wielomian $f \in \mathbb{F}[x_1, \dots x_n]$
można przedstawić w postaci:
$$
f = a_1f_1 + \dots + a_kf_k + r
$$
Na przykład dla $f(x, y) = x^2y + xy^2 + y^2$:
$$
f(x, y) = (x + y)(xy) + (y^2 - 1) + x + y + 1
$$

\section{Rozszerzony algorytm Euklidesa}

Dla $a, b \in \mathbb{Z}$ wyznacza $NWD(a, b)$ oraz $x, y \in
\mathbb{Z} : ax + by = NWD(a, b)$. Jest on zdefiniowany w następujący sposób:
$$
(r_0, s_0, t_0) = (a, 1, 0), (r_1, s_1, t_1) = (b, 0, 1)
$$
$$
(r_{i + 1}, s_{i + 1}, t_{i + 1}) = (r_{i - 1}, s_{i - 1}, t_{i - 1}) - \lfloor
\frac{r_{i - 1}}{r_i} \rfloor (r_i, s_i, t_i)
$$
Równanie $ax + by = c$ ma rozwiązanie w $\mathbb{Z}$ tylko jeśli
$NWD(a, b) | c$.
Na przykład: dla $30, 45$ mamy:
\begin{enumerate}
  \item (45, 1, 0), (30, 0, 1)
  \item (45, 1, 0) - 1 * (30, 0, 1) = (15, \textcolor{red}{1, -1})
  \item (30, 0, 1) - 2 * (15, 1, -1) = (0, -2, 3)
  \item $NWD(30, 45) = 15$
  \item $15 = -1 * 30 + 1 * 45$
\end{enumerate}
Albo inaczej: $61^{-1} \in \mathbb{Z}_{130} = ?$
$$
61^{-1} \in \mathbb{Z}_{130} \rightarrow 61x \equiv 1 \mod 130
\rightarrow 61x + 130y = 1
$$
\begin{enumerate}
  \item (130, 1, 0), (61, 0, 1)
  \item (130, 1, 0) - 2 * (61, 0, 1) = (8, 1, -2)
  \item (61, 0, 1) - 7 * (8, 1, -2) = (5, -7, 15)
  \item (8, 1, -2) - 1 * (5, -7, 15) = (3, 8, -17)
  \item (5, -7, 15) - 1 * (3, 8, -17) = (2, -15, 32)
  \item (3, 8, -17) - 1 * (2, -15, 32) = (1, \textcolor{red}{23, -49})
  \item (2, -15, 32) - 2 * (1, 23, -49) = (0, -61, 130)
  \item $NWD(61, 130) = 1$
  \item $1 = (-49) * 61 + 23 * 130$
\end{enumerate}

\section{Problem logarytmu dyskretnego}

Dane: $a, c \in \mathbb{Z}, n \in \mathbb{N}$. Cel: znaleźć $x \in
\mathbb{Z}_n$ takie, że $a^x = c \in \mathbb{Z}_n$. Alternatywnie można
zdefiniować postać ogólną, gdzie mamy grupę $G$ oraz $|G| \in \mathbb{P}$,
i chcemy znaleźć $x \in G : g^x = h$.

\section{Test na pierwszość Fermata}

Jeśli $p \in \mathbb{P}$ to $\forall_{a \in \mathbb{Z}_p \setminus
\{0\}} a^{p - 1} = 1 \in \mathbb{Z}_p$.

\begin{enumerate}
  \item Losujemy $a \in \mathbb{Z}_p \setminus \{0\}$
  \item Obliczamy $a^{p - 1} \mod p$
  \item Jeśli $a^{p - 1} \ne 1$ to $p$ nie jest liczbą pierwszą
\end{enumerate}

\noindent
Na przykład: $p = 7$, $a = 2$:
$$
2^{7 - 1} = 2^6 = 64 \mod 7 = 1
$$
Zatem $7$ może być liczbą pierwszą.

\noindent
Albo $p = 4$, $a = 2$:
$$
2^{4 - 1} = 2^3 = 8 \mod 4 = 0
$$
Zatem $4$ nie jest liczbą pierwszą.

\section{Twierdzenie Eulera}

Niech $\mathbb{Z}_n^\times = \{a \in \mathbb{Z}_n : NWD(a,n) = 1\}$,
$\varphi(n) = |\mathbb{Z}_n^\times|$. Dla każdego $a \in \mathbb{Z}_n^\times$:
$a^{\varphi(n)} \equiv 1 (\mod n)$.

Jeśli $n = p_1^{\alpha_1}p_2^{\alpha_2} \dots p_k^{\alpha_k}$ jest rozkładem
na czynniki pierwsze to:
$$
\varphi(n) = \prod_{i = 1}^k p_i^{\alpha_i - 1}(p_i - 1)
$$

\section{Chińskie twierdzenie o resztach}

Niech $m_1, \dots m_k \in \mathbb{N}$ będą parami względnie pierwsze
($NWD = 1$), oraz $M = \prod m$. Wtedy dla dowolnych $a_1, \dots a_k
\in \mathbb{Z}$ istnieje $x < M$ takie, że:
$$
x \equiv a_i \mod m_i
$$

Innymi słowy, układ kongruencji, gdzie kolejne $m_i$ są parami
względnie pierwsze, ma dokładnie jedno rozwiązanie w przedziale $[0, M)$.
Na przykład:
$$
\begin{cases}
  x \equiv 2 \mod 3\\
  x \equiv 3 \mod 4\\
  x \equiv 2 \mod 5
\end{cases}
$$
\begin{enumerate}
  \item Rozwiązaniem ogólnym pierwszej kongruencji jest: $2 + 3t$.
  \item $2 + 3t$ dla $t = 3$ rozwiązuje drugą kongruencję:
    $2 + 3 \cdot 3 =  \textcolor{red}{11} \equiv 3 \mod 4$.
  \item Zatem dwie powyższe kongruencje możemy zapisać jako $x \equiv
    \textcolor{red}{11} \mod 3 \cdot 4$ i jej rozwiązaniem jest $11 + 12t$.
  \item Wracamy teraz do kroku 2, czyli znajdujemy rozwiązanie
    trzeciej kongruencji: $11 + 12t \equiv 2 \mod 5$. $t = 3$ rozwiązuje tę
    kongruencję.
\end{enumerate}
Czyli $x = 11 + 12 \cdot 3 = 47$ jest rozwiązaniem wszystkich trzech
kongruencji.

\section{Faktoryzacja wielomianu nad ciałem skończonym}

Faktoryzacja danej liczby lub wielomianu to znalezienie takich
czynników, że ich iloczyn daje tę liczbę lub wielomian.

\subsection{Distinct-degree factorization}

Wielomian $f(x) = a_0 + a_1x^1 \dots$ nazywamy unormowanym jeśli $a_n = 1$.
Współczynniki $a_n$ nazywamy wiodącym. Ponieważ dla każdego $a \in
\mathbb{F}_q \setminus \{0\}$ mamy $a^{q-1}$ więc:
$$
x^q - x = \prod_{a \in \mathbb{F}_q} (x - a)
$$
Dla każdego $d \ge 1$, $x^{q^d} - x \in \mathbb{F}_q[x]$ jest
iloczynem wszystkich
nierozkładalnych unormowanych wielomianów w $\mathbb{F}_q[x]$ stopnia $k|d$.

\noindent
\textbf{Przykład:} Niech
$f(x) = x^{10} + x^8 + 2x^6 + 6x^5 + 5x^3 + 2x^2 + 6x + 4 \in \mathbb{Z}_7$.
\begin{enumerate}
  \item $h_0 = x, f_0 = f, q = 7$
  \item $h_1 = x^q = x^7, g_1 = NWD(h_1 - h_0, f_0) = NWD(x^7 - x,
    f_0) = \textcolor{green}{x^2 + 3x + 2}, f_1 = f_0 / g_1 = x^8 +
    4x^7 + x^6 + 3x^5 +
    5x^4 + 6x^3 + 2$
  \item $h_2 = h_1^q = 4x^6 + x^4 + 6x^3 + 3, g_2 = NWD(h_2 - x, f_1)
    = \textcolor{green}{x^2 + 1}, f_2 = f_1 / g_2 = x^6 + 4x^5 + 6x^3
    + 5x^2 + 2$
  \item $h_3 = x, g_3 = \textcolor{green}{x^6 + 4x^5 + 6x^3 + 5x^2 +
    2}, f_3 = \textcolor{red}{1}$
\end{enumerate}
Zatem $f(x) = g_1 \cdot g_2 \cdot g_3 = (x^2 + 3x + 2)(x^2 + 1)(x^6 +
4x^5 + 6x^3 + 5x^2 + 2)$.

\subsection{Algorytm Cantora i Zassenhausa}

Algorytm Cantora i Zassenhausa jest algorytmem probabilistycznym, który
służy do faktoryzacji wielomianów nad ciałami skończonymi. Dla wejściowego
wielomianu $f$ zwraca zbiór wielomianów $g_1, g_2, \dots, g_k$ takich, że
$$
f(x) = g_1(x)g_2(x) \dots g_k(x)
$$
Algorytm ten działa w następujący sposób:
\begin{enumerate}
  \item Losujemy $a \in \mathbb{F}_q$ i obliczamy $g = NWD(f, x^q - a)$.
  \item Jeśli $g$ jest nierozkładalny to zwracamy $g$.
  \item W przeciwnym razie dzielimy $f$ przez $g$ i powtarzamy krok 1.
\end{enumerate}

\section{Bazy Gröbnera}

Dla porządku $\prec$ na $\mathbb{Z^n}$ oraz $f_1 \dots f_n \in
\mathbb{F}[x_1, \dots x_n]$ to:
$$
\langle f_1, \dots, f_n \rangle = \{a_1f_1 + \dots + a_nf_n : a_i \in
\mathbb{F}[x_1, \dots, x_n]\}
$$
nazywamy idealem generowanym przez $f_1, \dots, f_n$. Skończony podzbiór ideału,
względem porządku $\prec$ nazywamy bazą Gröbnera, jeśli:
$$
\langle LT(g) : g \in G \rangle = \langle LT(f) : f \in I \rangle
$$

\end{document}