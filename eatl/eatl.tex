\documentclass{../notatki}

\title{Elementy algebry i teorii liczb}

\begin{document}

\tableofcontents

\noindent Kazali mi to zdawać, choć algebrę miałem jakbym nie miał ciekawszych
rzeczy do roboty i potrzebował tej powtórki. Fun times.

\section{Operacje}

Każdą funkcję która ma dwa argumenty i zwraca jeden wynik można nazwać
operacją. Teoretycznie zatem można konwencjonalne operatory traktować jako
funkcje. $+(1, 1)=2$

\subsection{Działanie wewnętrzne i zewnętrzne}

Działanie wewnętrzne w zbiorze $A$: $*: A \times A \rightarrow A$.
Działanie zewnętrzne w zbiorze $A$: $*: F \times A \rightarrow A$

\subsection{Własności operacji}

Rozróżniamy kilka własności, które mogą mieć operacje.

\begin{itemize}
  \item \textbf{Łączność} - $A * (B * C) = (A * B) * C$
  \item \textbf{Przemienność} - $A * B = B * A$
  \item \textbf{Rozdzielność} - $A * (B + C) = A * B + A * C$
  \item \textbf{Element neutralny} - $A * E = A$
  \item \textbf{Element odwrotny} - $A * A^{-1} = E$
\end{itemize}

\section{Grupa}

Grupa to zbiór $G$ z działaniem wewnętrznym $*$ jeśli:
\begin{itemize}
  \item $*$ jest łączne
  \item $*$ posiada element neutralny
  \item $*$ posiada element odwrotny
\end{itemize}
Dodatkowo jeśli $*$ jest przemienne to mamy grupę abelową.

\subsection{\texorpdfstring{Grupa $\mathbb{Z}_n$}{Grupa Zn}}

Specyficzna grupa, która jest zbiorem liczb całkowitych od $0$ do $n-1$ z
działaniem $+$ modulo $n$. Elementem przeciwnym dla $a$ jest $n - a$.

\subsection{\texorpdfstring{Grupa $\mathbb{Z}_n^\times$}{Grupa Znx}}

$$
\mathbb{Z}_n^\times = \{a \in \mathbb{Z}_n : NWD(a,n) = 1\}
$$
A działanie tej grupy to mnożenie modulo $n$. Element przeciwny oblicza się
algorytmem Euklidesa.

\section{Podgrupa}

Podgrupa to podzbiór grupy z odpowiednio dostosowanym działaniem. Na przykład
podgrupą $\mathbb{Z}_{12}$ jest $(\{0, 4, 8\}, +)$, ponieważ nie
ma pary elementów z
podzbioru, które po dodaniu dałyby coś spoza podzbioru.

\subsection{Generowanie}

Niech $(G, *)$ będzie grupą z elementem neutralnym $E$. Wtedy:
$$
\langle g \rangle = \{\stackrel{n}{\overbrace{g * g * \dots * g}} : n
\in \mathbb{N}\} \cup \{E\} \cup \{\stackrel{m}{\overbrace{g^{-1} * g^{-1} *
  \dots * g^{-1}}} : m
\in \mathbb{N}\}
$$
Jeśli $G = \langle g \rangle$ dla pewnego $g$ to $G$ jest grupą
cykliczną. Rzędem
$g$ jest $|\langle g \rangle|$

W $\mathbb{Z}_{12}$ podgrupą generowaną przez $4$ jest $\{0, 4, 8\}$, a
$rz(4) = |\langle 4 \rangle|$. Z kolei $\langle 1 \rangle = \mathbb{Z}_{12}$
zatem $\mathbb{12}$ jest grupą cykliczną. Jeżeli $p$ jest liczbą pierwszą
to $\mathbb{Z}_p^\times$ jest grupą cykliczną.

\section{Funkcja Eulera}

$$
\varphi(n) =
\begin{cases}
  1 : n = 1\\
  |\mathbb{Z}_n^\times| : n > 1\\
\end{cases}
$$

Jeśli $p$ jest liczbą pierwszą to $\varphi(p^k) = p^k - p^{k-1}$ oraz
$\varphi(p) = p - 1$. Jeśli $NWD(m, n) = 1$ to $\varphi(mn) =
\varphi(m)\varphi(n)$.

% z algebry

\section{Permutacje}

$$
\pi =
\begin{pmatrix}
  1   & 2   & 3   & \cdots & n-1   \\
  a_1 & a_2 & a_3 & \cdots &  a_n
\end{pmatrix}
$$

$$
a_n = \pi(n)
$$

\subsection{Rozkład na cykle}

$$
\pi =
\begin{pmatrix}
  a_1 & a_2 & a_3 & \cdots & a_n   \\
  a_2 & a_3 & a_4 & \cdots & a_1
\end{pmatrix}
= (a_1, a_2, a_3, \dots, a_n)
$$

$$
\pi' =
\begin{pmatrix}
  a_1 & a_2 & b_1 & b_2   \\
  a_2 & a_1 & b_2 & a_1
\end{pmatrix}
= (a_1, a_2) \cdot (b_1, b_2)
$$

\subsection{Iloczyn transpozycji}

$$
(a_1, a_2, a_3, \dots, a_k) = (a_1, a_k) \cdot (a_1, a_{k-1}) \cdot
\dots \cdot (a_1, a_3) \cdot (a_1, a_2)
$$

\subsection{Postać macierzowa}

$$
\pi =
\begin{pmatrix}
  1 & 2 & 3 & 4\\
  2 & 2 & 4 & 3
\end{pmatrix}
=
\begin{bmatrix}
  0 & 0 & 0 & 0\\
  1 & 1 & 0 & 0\\
  0 & 0 & 0 & 1\\
  0 & 0 & 1 & 0
\end{bmatrix}
$$

\subsection{Znak permutacji}

Ilość czynników w iloczynie transpozycji określa parzystość permutacji.

$$(-1)^n$$ gdzie $n$ to ilość transpozycji

\section{Pierścień}

Pierścień to uporządkowana trójka $R(A, +, \cdot)$, gdzie $A$ to
zbiór, a $+$ i $\cdot$ to działania spełniające następujące warunki:

\begin{itemize}
  \item $(A, +)$ jest grupą abelową
  \item $+$ i $\cdot$ są są wewnętrzne dla $A$
  \item Dla każdego $a, b, c \in A$ zachodzi rozdzielność mnożenia
    względem dodawania: $a \cdot (b + c) = a \cdot b + a \cdot c$
    oraz $(a + b) \cdot c = a \cdot c + b \cdot c$
  \item Istnieje element neutralny mnożenia $1 \in A: \forall a \in
    A: a \cdot 1 = 1 \cdot a = a$
\end{itemize}

\subsection{Pierścień z jedynką}

Pierścień z jedynką to pierścień, w którym istnieje element neutralny
mnożenia oraz $A \ne \emptyset$

\subsection{Pierścień przemienny}

Pierścień przemienny to pierścień, w którym mnożenie jest przemienna

\section{Ciało}

Ciało $\mathbb{C}(K, +, \cdot)$ to pierścień przemienny z jedynką, oraz $(K
\setminus \{0\}, \cdot)$ jest grupą.
% koniec algebry
Innymi słowy: jest to niepusty zbiór $K$ z działaniami $+$ i $\cdot$, które
są przemienne, łączne, posiadają elementy neutralne i odwrotne, oraz
istnieją takie pary $(a, b)$ dla których:
$$
a + b = 0 \text{ oraz } a \cdot b = 1
$$
Przykładami ciał są: $\mathbb{Q}$, $\mathbb{R}$, $\mathbb{C}$.

\section{Wielomiany}

Mówimy, że liczba $z$ jest pierwiastkiem $n$-tego stopnia liczby $w$ jeśli
$$
z^n = w
$$
Każdy wielomian $f \in \mathbb{C}[x]$ stopnia $n$ ma $n$
pierwiastków. Jeśli $f(x) = a_n x^n + a_{n-1} x^{n-1} + \dots + a_0$ to
$$
f(x) = a_n (x - z_1)(x - z_2) \dots (x - z_n)
$$

\subsection{Przykład ciała wielomianowego}

Zbiór $\{0, 1, x, x+1\}$ z dodawaniem i mnożeniem modulo $f(x) = x^2
+ x + 1 \in \mathbb{Z}_2[x]$ jest ciałem.

\begin{table*}[h]
  \centering
  \begin{tabular}{c|cccc}
    \hline
    $+$ & 0 & 1 & $x$ & $x + 1$ \\
    \hline
    0 & 0 & 1 & $x$ & $x + 1$ \\
    1 & 1 & 0 & $x + 1$ & $x$ \\
    $x$ & $x$ & $x + 1$ & 0 & 1 \\
    $x + 1$ & $x + 1$ & $x$ & 1 & 0 \\
    \hline
  \end{tabular}
  \caption{Dodawanie w wyżej zdefiniowanym ciele}
\end{table*}

\subsection{Rozkładalność a ciała}

Dlaczego zbiór $\{0, 1, x, x + 1\}$ z dodawaniem i mnożeniem modulo
$f(x) = x^2 + 1 \in \mathbb{Z}_2[x]$ nie jest ciałem? Ponieważ $x^2 +
1$ jest rozkładalny w $\mathbb{Z}_2[x]$.

Mówimy, że wielomian $f(x)$ jest rozkładalny w $\mathbb{Z}_p[x]$
jeśli gdy istnieją
wielomiany $g_1, g_2 \in \mathbb{Z}_p[x]$ stopnia co najmniej $1$ takie, że
$f(x) = g_1(x)g_2(x)$.

Dla każdego $n \in \mathbb{N}$ i każdej liczby pierwszej $p$ istnieje wielomian
stopnia $n$ w $\mathbb{Z}_p[x]$ który jest nierozkładalny.

\end{document}