\documentclass{../notatki}

\title{Zadania z elektryczności i magnetyzmu}

\begin{document}

\begin{enumerate}
  \item Proton porusząjący się z prędkością $20 \frac{m}{s}$ wpada w
    pole magnetyczne i zaczyna poruszać się po okręgu z prędkością
    kołową $0.5 Hz$. Jaka jest wartość indukcji pola magnetycznego?
    Dana jest masa protonu $m_p$, oraz ładunek protonu $q_p$.
  \item W rogach trójkąta równobocznego o boku $a$ umieszczono
    jednoimienne ładunki $q$. Jaki ładunek $Q$ trzeba by umieścić w
    środku trójkąta, aby zrównoważyć siły?
  \item Elektron porusza się swobodnie w próżni w polu elektrycznym
    kondensatora płaskiego, podłączonego do źródła napięcia $U =
    12V$. Jaką prędkość uzyska elektron po przebyciu połowy
    odległości między płytkami? Dana jest masa elektronu $m_e$ oraz
    ładunek elektronu $q_e$.
  \item Znajdź pojemności zastępcze układu kondensatorów o pojemnościach $C$
    między punktami $AB, BC, AC$.
    \begin{figure*}[h]
      \centering
      \begin{tikzpicture}
        \draw (0,4) to[C] (0,0);
        \draw (0,4) to[C] (2,4) to[C] (2,0) to[C] (0,0);
        \draw (2,4) to[C] (4,4) to[C] (4,0) to[C] (2,0);

        \draw (2,4) node[anchor=south] {A};
        \draw (2,0) node[anchor=north] {B};
        \draw (4,0) node[anchor=north] {C};
      \end{tikzpicture}
    \end{figure*}

  \item Jaka musi być indukcja pola magnetycznego (kierunek i wartość) aby pręt
    miedziany o średnicy przekroju $2 mm$, umieszczony poziomo zaczął lewitować
    kiedy przepuścimy przez niego prąd $I = 2A$. Dana jest gęstość
    miedzi.
  \item Okrągła miedziana ramka o promieniu $10 cm$ zbudowana z drutu o grubości
    $2 mm$ znajduje się w polu magnetycznym prostopadle do linii sił
    pola. Ile musi
    wynosić indukcja pola magnetycznego, żeby w ramce przepłynął
    ładunek $0.1 C$? Dany jest opór elektryczny miedzi $\rho$.
\end{enumerate}

\end{document}