\documentclass{../notatki}

\title{Grafika komputerowa}

\begin{document}

\tableofcontents

\section{Literatura}

\begin{itemize}
    \item \href{https://wp.faculty.wmi.amu.edu.pl/GRK.html}{Wykład}
    \item \href{https://learnopengl.com/book/book_pdf.pdf}{J. de Vries, "Learn OpenGL"}
    \item \href{https://andkok.faculty.wmi.amu.edu.pl/grk/zad_1/}{Ćwiczenia}
\end{itemize}

\section{Wektory}

$$
\vec{v} \cdot \vec{w} = v_x \cdot w_x + v_y \cdot w_y + v_z \cdot w_z
$$

$$
\vec{v} \cdot \vec{w} = |\vec{v}| \cdot |\vec{w}| \cdot cos(\theta)
$$

Jeśli $\vec{v}$ i $\vec{w}$ są prostopadłe to $\vec{v} \cdot \vec{w} = 0$.
A z kolei jeśli $\vec{v} \cdot \vec{w}$ są normalne to $\vec{v} \cdot \vec{w} = cos(\theta)$.

\section{Operacje}

Wyróżniamy trzy rodzaje operacji na punktach w przestrzeni n-wymiarowej:

\begin{itemize}
    \item Skalowanie\\
    $$
    f(
        \begin{bmatrix}
        x \\
        y
        \end{bmatrix}
    )
    =
    \begin{bmatrix}
        s_x * x \\
        s_y * y
    \end{bmatrix}
    $$
    \item Rotacja\\
    $$
    f(
        \begin{bmatrix}
        x \\
        y
        \end{bmatrix}
    )
    =
    \begin{bmatrix}
        x * cos(\theta) - y * sin(\theta) \\
        x * sin(\theta) + y * cos(\theta)
    \end{bmatrix}
    $$
    \item Translacja\\
    $$
    f(
        \begin{bmatrix}
        x \\
        y
        \end{bmatrix}
    )
    =
    \begin{bmatrix}
        x + 1 \\
        y - 1
    \end{bmatrix}
    $$
\end{itemize}

\section{Operacje 2D}

\subsection{Skalowanie}

$$
S = \begin{bmatrix}
        s_x & 0 \\
        0 & s_y
    \end{bmatrix}
\text{lub }
\begin{bmatrix} s_x & 0 & 0 \\ 0 & s_y & 0 \\ 0 & 0 & 1 \end{bmatrix}
$$

\subsection{Rotacja}

$$
R = \begin{bmatrix}
        cos(\theta) & -sin(\theta) \\
        sin(\theta) & cos(\theta)
    \end{bmatrix}
\text{lub }
\begin{bmatrix} cos(\theta) & -sin(\theta) & 0 \\ sin(\theta) & cos(\theta) & 0 \\ 0 & 0 & 1 \end{bmatrix}
$$

\subsection{Translacja}

$$
T = 
\begin{bmatrix}
    1 & 2 \\
    3 & 4
\end{bmatrix}
\text{lub dla } 
\vec{V}=(T_x, T_y),
T=
\begin{bmatrix} 1 & 0 & T_x \\ 0 & 1 & T_y \\ 0 & 0 & 1 \end{bmatrix}
$$

\section{Konkatenacja operacji}

Operacje można łączyć w łańcuchy, np. $T * R * S$
Jako że to jest mnożenie macierzy, to kolejność ma znaczenie.
Najpierw wykonuje się operację z prawej strony.

\section{Operacje 3D}

\subsection{Skalowanie}

$$
S = \begin{bmatrix} s_x & 0 & 0 & 0 \\ 0 & s_y & 0 & 0 \\ 0 & 0 & s_z & 0 \\ 0 & 0 & 0 & 1 \end{bmatrix}
$$

\subsection{Rotacja}

$$
R_x(\theta) = 
\begin{bmatrix}
    1 & 0 & 0 \\
    0 & cos(\theta) & -sin(\theta) \\
    0 & sin(\theta) & cos(\theta) 
\end{bmatrix}
$$

$$
R_y(\theta) = 
\begin{bmatrix}
    cos(\theta) & 0 & sin(\theta) \\
    0 & 1 & 0 \\
    -sin(\theta) & 0 & cos(\theta) 
\end{bmatrix}
$$

$$
R_z(\theta) = 
\begin{bmatrix}
    cos(\theta) & -sin(\theta) & 0 \\
    sin(\theta) & cos(\theta) & 0 \\
    0 & 0 & 1
\end{bmatrix}
$$

\subsection{Translacja}

$$
\text{Dla wektora }
\vec{V}=(T_x, T_y, T_z),
T=
\begin{bmatrix} 1 & 0 & 0, & T_x \\ 0 & 1 & 0 & T_y \\ 0 & 0 & 1 & T_z \\ 0 & 0 & 0 & 1 \end{bmatrix}
$$

\section{Kamery}

Podczas patrzenia na obiekt w polu widzenia kamery można wyróżnić następujące płaszczyzny:

\begin{itemize}
    \item Górna
    \item Dolna
    \item Lewa
    \item Prawa
    \item Bliska
    \item Daleka
\end{itemize}

\subsection{Projekcje}

Kluczowym aspektem kamery jest projekcja, czyli rzutowanie obiektów na płaszczyznę ekranu.
Trzeba w jakiś sposób zamienić punkty w 3D na punkty w 2D.

\subsubsection{Ortogonalna}

\begin{itemize}
    \item Wszystkie linie są równoległe
    \item Brak perspektywy
    \item Obiekty są przeskalowane
    \item Nie ma odczucia głębi, obiekty pozostają tego samego rozmiaru niezależnie od odległości
    \item Linie równoległe pozostają równoległe po projekcji
\end{itemize}

\subsubsection{Perspektywiczna}

\begin{itemize}
    \item Linie zbiegają się w jednym punkcie
    \item Obiekty są przeskalowane
    \item Odczucie głębi
    \item Obiekty są mniejsze w miarę oddalania się od kamery
    \item "Normalna" projekcja
\end{itemize}

\section{Pipeline}

Renderowanie obrazu składa się z kilku etapów:

$$
\text{Object Space} \rightarrow \text{World space} \overset{M_{cam}}{\rightarrow} \text{View space} \overset{M_{pro}}{\rightarrow} \text{Clip space} \overset{M_{vp}}{\rightarrow} \text{Screen space}
$$

\subsection{VP}

Dla widoku wielkości $n_x * n_y$
$$
M_{vp} = \begin{bmatrix} 
    \frac{n_x}{2} & 0 & 0 & \frac{n_x - 1}{2} \\
    0 & \frac{n_y}{2} & 0 & \frac{n_y - 1}{2} \\
    0 & 0 & 1 & 0 \\
    0 & 0 & 0 & 1 
\end{bmatrix}
$$

\subsection{Pro}

Dla przestrzeni widoku ortogonalnego stworzonego przez trzy punkty:

- $(l,b,n)$ w lewym dolnym rogu
- $(r,b,n)$ w prawym dolnym rogu
- $(r,t,f)$ w prawym górnym rogu

$$
M_{ort} = \begin{bmatrix} 
    \frac{2}{r - l} & 0 & 0 & -\frac{l + r}{r - l} \\
    0 & \frac{2}{t - b} & 0 & -\frac{b + t}{t - b} \\
    0 & 0 & \frac{2}{n - f} & -\frac{n + f}{n - f} \\
    0 & 0 & 0 & 1 
\end{bmatrix}
$$

$$
M_{per} = M_{ort} \cdot \begin{bmatrix} 
    n & 0 & 0 & 0 \\
    0 & n & 0 & 0 \\
    0 & 0 & n + f & -nf \\
    0 & 0 & 0 & 1 
\end{bmatrix}
$$

\subsection{Cam}

Dla wektorów określających kierunek patrzenia kamery: $\vec{g}$, $\vec{v}$, $\vec{u}$, $\vec{w} = -v$ oraz $\vec{e}$ określającym przesunięcie kamery

$$
M_{cam} = \begin{bmatrix} 
    u_x & u_y & u_z & 0 \\
    v_x & v_y & v_z & 0 \\
    w_x & w_y & w_z & 0 \\
    0 & 0 & 0 & 1 
\end{bmatrix}
\begin{bmatrix}
    1 & 0 & 0 & -e_x \\
    0 & 1 & 0 & -e_y \\
    0 & 0 & 1 & -e_z \\
    0 & 0 & 0 & 1 
\end{bmatrix}
$$

\subsection{Model}

Matryca modelu to po prostu macierz transformacji obiektu.
Jak obiekt jest obrócony, przesunięty, skalowany.
To zależy od tego gdzie jest obiekt w przestrzeni, jako że zwykle punkty modelu są w przestrzeni obiektu (relatywnie do środka obiektu).

\subsection{Złożenie}

$$
pipeline(\begin{bmatrix}x \\ y \\ z\\ 1\end{bmatrix}) = M_{vp}M_{ort}M_{cam}\begin{bmatrix}x \\ y \\ z\\ 1\end{bmatrix}
$$

\section{Algorytmy Renderowania}

Jako, że w procesie przekształcenia pipeline zatracana jest informacja o koordynacie $z$ punktów, konieczne jest renderowanie obiektów w konkretnej kolejności.

\subsection{Algorytm malarza}

Ten algorytm jest niezwykle prosty.
Obiekty się sortuje wobec pozycji w $z$ i najpierw renderuje się te w tle.
Niestety, nie działa to dla obiektów przecinających się i innych o skomplikowanej topologii.

\subsection{Z-buffer}

Dla każdego obiektu zapisujesz na jakim z "jest" pixel i jakiego koloru jest.
Jeśli w globalnym z bufferze ten pixel ma większe z to w bufferze pixel jest zastępowany.
Efektem działania algorytmu jest buffer z pixelami "renderowany od tyłu".

\section{Culling i Clipping}

W procesie renderowania koniecznym jest czasami obsługiwanie co jeśli obiekt wychodzi poza kąt widzenia oraz co jeśli obiekt jest niewidoczny.

\subsection{Clipping}

Czyli obcinanie obiektów składa się z dwóch etapów:

\begin{itemize}
    \item Wykrywanie czy obiekt przecina jedną z płaszczyzn
    \item Tworzenie nowych mniejszych obiektów
\end{itemize}

\subsubsection{Wykrywanie}

Jeśli dla n=normalny wektor płaszczyzny oraz q=punkt na płaszczyźnie $f(p) = n * (p - q) < 0$ to punkt p jest wewnątrz.

\subsubsection{Tworzenie}

Dla każdej krawędzi obiektu sprawdzamy czy przecina ona płaszczyznę.
Dzieląc trójkąty na mniejsze wewnątrz pola widzenia, tworzymy nowe trójkąty na podstawie punktów przecięcia z płaszczyzną.

\subsection{Culling}

Odrzucanie trójkątów których wektor normalny jest skierowany od punktu widzenia to backface culling.
Any zobaczyć czy wektor normalny $\vec{n}$ jest skierowany od punktu widzenia to wystarczy złożyć go z wektorem punktu widzenia $\vec{v}$
Jeśli $\vec{n} \cdot \vec{v} < 0$ to trójkąt jest skierowany do kamery.

\section{Oświetlenie}

$$
I = f_{att}(D + S) + A
$$

\subsection{Diffuse}

$$
D = I_p \cdot K_d \cdot max(0, cos(\theta))
$$

gdzie:

\begin{itemize}
    \item $I_p$ to intensywność światła
    \item $K_d$ to współczynnik odbicia
    \item $\theta$ to kąt między wektorem światła a wektorem normalnym do powierzchni
\end{itemize}

\subsection{Specular}

$$
S = I_p \cdot K_s \cdot max(0, cos(\alpha))^n
$$

gdzie:

\begin{itemize}
    \item $I_p$ to intensywność światła
    \item $K_s$ to współczynnik odbicia
    \item $\alpha$ to kąt między wektorem do kamery a wektorem odbicia
    \item $n$ to współczynnik tłumienia
\end{itemize}

\subsection{Ambient}

$$
A = I_a \cdot K_a
$$

gdzie:

\begin{itemize}
    \item $I_a$ to intensywność światła
    \item $K_a$ to współczynnik odbicia
\end{itemize}

\subsection{Attenuation}

$$
f_{att} = 1 - \frac{d}{r}^2
$$

gdzie:

\begin{itemize}
    \item $d$ to odległość od źródła światła
    \item $r$ to promień światła
\end{itemize}

\subsection{Shading}

\begin{itemize}
    \item Flat shading - kolor obiektu jest jednolity
    \item Gouraud shading - kolor obiektu jest interpolowany między wierzchołkami
    \item Phong shading - kolor obiektu jest interpolowany między pikselami
\end{itemize}

\end{document}
