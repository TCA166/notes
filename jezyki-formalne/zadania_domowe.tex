\documentclass{../notatki}

\title{Zadania domowe z Języków Formalnych}

\begin{document}

\section{Znajdź języki}

\begin{itemize}
  \item $L_1(L_1 \setminus  L_2) \ne L_1 $
  \item $(L_1 \cup L_2)^* \ne L_1^* \cup L_2^*$
\end{itemize}

\divider
Dla $L_1 = \{a^n : n > 1\}$ oraz $L_2 = \{b^n : n > 1\}$:

\begin{itemize}
  \item $L_1 \setminus L_2 = L_1$, $L_1 \cdot L_1 \ne L_1$, zatem
    pierwsza nierówność jest spełniona.
  \item $L_1^* = L_1 $ i $L_2^* = L_2$, z kolei $"ba" \in (L_1 \cup
    L_2)^*$ oraz $"ba" \notin (L_1 \cup L_2)$ zatem $(L_1 \cup L_2)^*
    \ne L_1^* \cup L_2^*$
\end{itemize}
Dla powyższych języków równości nie są prawdziwe.

\section{}

$$
L_1 = \{a^nb^m : n,m \ge 1\}, L_2 = \{a, b, c\}
$$
\subsection{Oblicz 0,1,2,3,4 potęgi języków}
$L_1^0 = \{\epsilon\}$, $L_1^1 = L_1$\\
$L_1^2 = \{a^nb^ma^ob^p : n,m,o,p \ge 1\}$\\
$L_1^3 = \{a^nb^ma^ob^pa^qb^r : n,m,o,p,q,r \ge 1\}$\\
$L_1^4 = \{a^nb^ma^ob^pa^qb^ra^sb^t : n,m,o,p,q,r,s,t \ge 1\}$\\
$L_2^0 = \{\epsilon\}$, $L_2^1 = L_2$\\
$L_2^2 = \{P \in \{a,b,c\}^* : 2| |P|\}$\\
$L_2^3 = \{P \in \{a,b,c\}^* : 3| |P|\}$\\
$L_2^4 = \{P \in \{a,b,c\}^* : 4| |P|\}$\\

\subsection{Podaj słowny opis języków}

\begin{itemize}
  \item $L_1$ - język składający się ze słów złożonych z co najmniej
    jednej litery $a$ i co najmniej jednej litery $b$.
  \item $L_2$ - język składający się z pojedynczych liter $a$, $b$ i $c$.
\end{itemize}

\section{Narysuj diagram automatu skończenie stanowego, który
akceptuje język}

$$
L(\mathfrak{A}) = \{P\ \in \{a, b, c\}^*: aabc \subset P\}
$$

\divider

\begin{figure*}[h]
  \centering
  \begin{tikzpicture}[shorten >=1pt,node distance=2.0cm,on grid,auto]
    \node[state,initial] (q_0)   {$q_0$};
    \node[state](q_1) [right=of q_0] {$q_1$};
    \node[state](q_2) [right=of q_1] {$q_2$};
    \node[state](q_3) [right=of q_2] {$q_3$};
    \node[state, accepting](q_4) [right=of q_3] {$q_4$};
    \path[->]
    (q_0) edge [bend left] node {a} (q_1)
    edge  [loop above] node {b,c} ()
    (q_1) edge  [bend left] node {a} (q_2)
    edge  [bend left] node {b,c} (q_0)
    (q_2) edge  [loop above] node {a} ()
    edge  [bend left] node {b} (q_3)
    edge  [bend left] node[swap] {c} (q_0)
    (q_3) edge  [bend left] node {c} (q_4)
    edge  [bend left] node[swap] {a} (q_1)
    edge [bend left] node {b} (q_0)
    (q_4) edge  [loop above] node {a,b,c} ();
  \end{tikzpicture}
\end{figure*}

\section{Narysuj diagram automatu skończenie stanowego, który
akceptuje język}

$$
L(\mathfrak{A}) = \{P\ \in \{a, b, c\}^*: P'aabc, P'
\in \{a, b, c\}^*\}
$$

\divider

\begin{figure*}[h]
  \centering
  \begin{tikzpicture}[shorten >=1pt,node distance=2.0cm,on grid,auto]
    \node[state,initial] (q_0)   {$q_0$};
    \node[state](q_1) [right=of q_0] {$q_1$};
    \node[state](q_2) [right=of q_1] {$q_2$};
    \node[state](q_3) [right=of q_2] {$q_3$};
    \node[state, accepting](q_4) [right=of q_3] {$q_4$};
    \path[->]
    (q_0) edge [bend left] node {a} (q_1)
    edge  [loop above] node {b,c} ()
    (q_1) edge  [bend left] node {a} (q_2)
    edge  [bend left] node {b,c} (q_0)
    (q_2) edge  [loop above] node {a} ()
    edge  [bend left] node {b} (q_3)
    edge  [bend left] node[swap] {c} (q_0)
    (q_3) edge  [bend left] node {c} (q_4)
    edge  [bend left] node {a} (q_1)
    edge [bend left] node {b} (q_0)
    (q_4) edge [bend left] node[swap] {a} (q_1)
    edge [bend left] node {b,c} (q_0);
  \end{tikzpicture}
\end{figure*}

\section{Przeprowadź determinizacje automatu}

\begin{figure*}[h]
  \centering
  \begin{tikzpicture}[shorten >=1pt,node distance=2.0cm,on grid,auto]
    \node[state,initial] (q_0)   {$q_0$};
    \node[state, accepting](q_1) [right=of q_0] {$q_1$};
    \node[state](q_2) [below right=of q_0] {$q_2$};
    \path[->]
    (q_0) edge [bend left] node {$\epsilon$} (q_1)
    edge [bend right] node {a} (q_2)
    (q_1) edge  [bend left] node {a} (q_0)
    (q_2) edge  [loop right] node {b} ()
    edge [bend right] node[midway, right] {a,b} (q_1);
  \end{tikzpicture}
\end{figure*}

\begin{table}[h!]
  \centering
  \begin{tabular}{c|c|c|c}
    $\delta$          & a         & b              & $\epsilon$ \\ \hline
    $\rightarrow q_0$ & $\{q_2\}$ & $\emptyset$    & $\{q_1\}$ \\ \hline
    $\underline{q_1}$ & $\{q_0\}$ & $\emptyset$    & $\emptyset$ \\ \hline
    $q_2$             & $\{q_1\}$ & $\{q_1, q_2\}$ & $\emptyset$ \\
  \end{tabular}
\end{table}

\divider

\begin{table}[h!]
  \centering
  \begin{tabular}{c|c|c}
    $\delta$                      & a              & b \\ \hline
    $\rightarrow q_0$             & $\{q_2\}$      & $\emptyset$ \\ \hline
    $\rightarrow \underline{q_1}$ & $\{q_0, q_1\}$ & $\emptyset$\\ \hline
    $q_2$                         & $\{q_1\}$      & $\{q_1, q_2\}$ \\
  \end{tabular}
\end{table}

$$
\mathfrak{A} = \langle \{q^\emptyset, q^0, q^1, q^2, q^{01}, q^{12},
q^{02}, q^{012}\},\{a, b\},\delta,q^{01},H = \{q^1, q^{01}, q^{12},
q^{012}\} \rangle
$$

\begin{table}[h!]
  \centering
  \begin{tabular}{c|c|c}
    $\delta$                       & a              & b \\ \hline
    $q^\emptyset$                  & $q^\emptyset$  & $q^\emptyset$ \\ \hline
    $q^0$                          & $q^2$          & $q^\emptyset$ \\ \hline
    $\underline{q^1}$              & $q^{01}$       & $q^\emptyset$ \\ \hline
    $q^2$                          & $q^1$          & $q^{12}$ \\ \hline
    $\rightarrow\underline{q^{01}}$& $q^{012}$      & $q^\emptyset$ \\ \hline
    $\underline{q^{12}}$           & $q^{01}$       & $q^{12}$ \\ \hline
    $q^{02}$                       & $q^{12}$       & $q^{12}$ \\ \hline
    $\underline{q^{012}}$          & $q^{012}$      & $q^{12}$ \\
  \end{tabular}
\end{table}

\section{Narysuj diagram automatu skończenie stanowego, który
akceptuje język}

$$
L(\mathfrak{A}) = \{P\ \in \{a, b, c\}^* \wedge (ab)^n \nsubseteq P
: n \ge 3\}
$$

\begin{table}[h!]
  \centering
  \begin{tabular}{c|c|c|c}
    $\delta$                       & a           & b            & c \\ \hline
    $\underline{\rightarrow q_0}$  & $q_a$       & $q_b$        &
    $q_c$ \\ \hline
    $\underline{q_a}$              & $q_a$       & $q_{ab}$     &
    $q_c$ \\ \hline
    $\underline{q_b}$              & $q_a$       & $q_b$        &
    $q_c$ \\ \hline
    $\underline{q_c}$              & $q_a$       & $q_b$        &
    $q_C$ \\ \hline
    $\underline{q_{ab}}$           & $q_{aba}$   & $q_b$        &
    $q_c$ \\ \hline
    $\underline{q_{aba}}$          & $q_a$       & $q_{abab}$   &
    $q_c$ \\ \hline
    $\underline{q_{abab}}$         & $q_{ababa}$ & $q_b$        &
    $q_c$ \\ \hline
    $\underline{q_{ababa}}$        & $q_a$       & $q_f$        &
    $q_c$ \\ \hline
    $q_f$                          & $q_f$       & $q_f$        &
    $q_f$ \\ \hline
  \end{tabular}
\end{table}

\section{Stwórz automat akceptujący język generowany przez gramatykę}

$$
F = \{S \rightarrow \epsilon, S \rightarrow aB, B \rightarrow bA, B
\rightarrow bS, A \rightarrow aA, A \rightarrow \epsilon\}
$$

\divider

\begin{figure*}[h]
  \centering
  \begin{tikzpicture}[shorten >=1pt,node distance=2cm,on grid,auto]
    \node[state,initial, accepting] (q_0)   {$q_S$};
    \node[state](q_1) [right=of q_0] {$q_B$};
    \node[state, accepting](q_2) [right=of q_1] {$q_A$};
    \path[->]
    (q_0) edge [bend left] node {a} (q_1)
    (q_1) edge node {b} (q_2)
    edge [bend left] node {b} (q_0)
    (q_2) edge  [loop right] node {a} ();
  \end{tikzpicture}
\end{figure*}

$$
L(G) = L(\mathfrak{A}) = L(\epsilon + (ab)^* + (ab)^*ab(a)^*) =
L((ab)^*(\epsilon + ab(a)^*))
$$

\end{document}