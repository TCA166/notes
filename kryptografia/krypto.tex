\documentclass{../notatki}

\title{Kryptografia z elementami algebry}

\begin{document}

\tableofcontents

\section{Szyfr Shannona}

Szyfr według Shannon'a jest zdefiniowany jako:
$$
\pi = (E, D) : (C, M, K)
$$
gdzie schemat szyfrujący $E$ i schemat deszyfrowania $D$ są funkcjami:
$$
E: M \times K \rightarrow C
$$
$$
D: C \times K \rightarrow M
$$

$$
D(k, E(k, m)) = m
$$

\subsection{Szyfr XOR}

$$
K = M = C = \{0, 1\}^L
$$

$$
E(m, k) = m \oplus k
$$
$$
D(c, k) = c \oplus k
$$

\subsection{Bezpieczeństwo doskonałe}

Niech $\pi$ będzie szyfrem Shannona. Rozważmy eksperyment losowy, w którym
zmienna losowa $K$ ma rozkład jednostajny nad $K$. Jeśli zachodzi:
$$
\forall_{m_0, m_1 \in M} \forall_{c \in C} P(E(k, m_0) = c) = P(E(k, m_1) = c)
$$
to mówimy, że szyfr $\pi$ jest szyfrem doskonałym.

Jeśli $\pi$ jest szyfrem doskonałym, to $|K| \ge |M|$.

\section{Struktury algebraiczne}

\begin{enumerate}
  \item $\forall_{a, b \in G} a * (b * c) = (a * b) * c$
  \item $\forall_{a, b \in G} a * b = b * a$
  \item $\exists_{e \in G} \forall_{a \in G} a * e = a$
  \item $\forall_{a \in G} a^{-1} = e$
\end{enumerate}

\begin{itemize}
  \item półgrupa: 1
  \item monoid: 1, 3
  \item grupa: 1, 3, 4
  \item grupa abelowa: 1, 2, 3, 4
\end{itemize}

Zawsze istnieje tylko jeden element neutralny operacji.
Rzędem grupy jest moc zbioru $G$.

$$\varphi(n) = |\{a \in \mathbb{Z}_n : \gcd(a, n) = 1\}|$$

\subsection{Podgrupa}

Niech $H$ będzie podgrupą grupy $G$. Wtedy:
$$
\forall_{a, b \in H} a * b \in H
$$
$$
\forall_{a \in H} a^{-1} \in H
$$
Na przykład, dla $\mathbb{Z}_{10} = \{0, 1, 2, 3, 4, 5, 6, 7, 8, 9\}$,
$H = \{0, 2, 4, 6, 8\}$ jest podgrupą grupy $\mathbb{Z}_{10}$.

\subsection{Generatory}

$$
\langle g \rangle = \{g^k : k \in \mathbb{Z}\}
$$

Grupa cykliczna, to grupa, która posiada co najmniej jednoelementowy zbiór
generatorów. $\exists_{g \in G} \langle g \rangle = G$

\subsection{Problem logarytmu dyskretnego}

Niech $G = \langle g \rangle$. Problemem jest znalezienie $x$
takiego, że $g^x = a$. W zależności od grupy oraz jej rozmiaru,
ten problem może być niezwykle trudny.

\subsection{Warstwy}

Dla podgrupy $H$ grupy $G$, warstwą lewostronną $H$ wyznaczoną przez
$a \in G$ jest zbiór:
$$
\begin{cases}
  a + H = \{a + h : h \in H\}
  aH = \{ah : h \in H\}
\end{cases}
$$

Warstwy są identyczne, albo rozłączne. Suma mnogościowa warstw jest równa
grupie $G$. Indeksem podgrupy $H$ w grupie $G$ ($G : H$) nazywamy moc zbioru
warstw względem podgrupy $H$.
$$
G : H = \frac{|G|}{|H|}
$$
Rząd podgrupy $H$ jest dzielnikiem rzędu grupy $G$.

\section{RSA}

Asymetryczny algorytm szyfrujący, w którym każda strona ma parę
kluczy: publiczny i prywatny. Enkrypcja odbywa się przy pomocy klucza
publicznego drugiej strony, a dekrypcja przy pomocy klucza prywatnego.

\subsection{Definicja}

Dla danych liczb pierwszych $p$ i $q$.

$$
n = pq
$$
$$
\varphi(n) = (p-1)(q-1)
$$
Następnie wybieramy liczbę $e$ względnie pierwszą z $\varphi(n)$.
Klucz prywatny $d$ musi spełniać warunek $ed \equiv 1 \pmod{\varphi(n)}$, zatem
$$
d = e^{-1} \pmod{\varphi(n)}
$$
$(n, e)$ tworzy klucz publiczny, a $(n, d)$ klucz prywatny.

Szyfrowanie wiadomości $M$ odbywa się za pomocą wzoru:
$$
C = M^e \pmod{n}
$$
Odkrycie wiadomości $M$ odbywa się za pomocą wzoru:
$$
M = C^d \pmod{n}
$$

\subsection{Trudność problemu}

Trudność wynika ze znalezienia $\varphi(n)$, a ponieważ weryfikacja
czy znalezione $\varphi(n)$ jest poprawne wymaga zastosowania rozszerzonego
algorytmu Euklidesa; odszyfrowanie wiadomości $C$ wymaga znalezienia $d$.

\subsection{Przykład}

$$
p = 7, q = 11 \Rightarrow n = 77, \varphi(n) = 60
$$
$$
e = 13 \Rightarrow d = 37 \Rightarrow
\begin{cases}
  (n, e) = (77, 13) \\
  (n, d) = (77, 37)
\end{cases}
$$

$$
M = 15 \Rightarrow C = 15^{13} \pmod{77} = 64
$$
$$
C = 64 \Rightarrow M = 64^{37} \pmod{77} = 15
$$

\end{document}
