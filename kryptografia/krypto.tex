\documentclass{../notatki}

\title{Kryptografia z elementami algebry}

\begin{document}

\tableofcontents

\section{RSA}

Asymetryczny algorytm szyfrujący, w którym każda strona ma parę
kluczy: publiczny i prywatny. Enkrypcja odbywa się przy pomocy klucza
publicznego drugiej strony, a dekrypcja przy pomocy klucza prywatnego.

\subsection{Definicja}

Dla danych liczb pierwszych $p$ i $q$.

$$
n = pq
$$
$$
\varphi(n) = (p-1)(q-1)
$$
Następnie wybieramy liczbę $e$ względnie pierwszą z $\varphi(n)$.
Klucz prywatny $d$ musi spełniać warunek $ed \equiv 1 \pmod{\varphi(n)}$, zatem
$$
d = e^{-1} \pmod{\varphi(n)}
$$
$(n, e)$ tworzy klucz publiczny, a $(n, d)$ klucz prywatny.

Szyfrowanie wiadomości $M$ odbywa się za pomocą wzoru:
$$
C = M^e \pmod{n}
$$
Odkrycie wiadomości $M$ odbywa się za pomocą wzoru:
$$
M = C^d \pmod{n}
$$

\subsection{Trudność problemu}

Trudność wynika ze znalezienia $\varphi(n)$, a ponieważ weryfikacja
czy znalezione $\varphi(n)$ jest poprawne wymaga zastosowania rozszerzonego
algorytmu Euklidesa; odszyfrowanie wiadomości $C$ wymaga znalezienia $d$.

\subsection{Przykład}

$$
p = 7, q = 11 \Rightarrow n = 77, \varphi(n) = 60
$$
$$
e = 13 \Rightarrow d = 37 \Rightarrow
\begin{cases}
  (n, e) = (77, 13) \\
  (n, d) = (77, 37)
\end{cases}
$$

$$
M = 15 \Rightarrow C = 15^{13} \pmod{77} = 64
$$
$$
C = 64 \Rightarrow M = 64^{37} \pmod{77} = 15
$$

\end{document}
