\documentclass{../notatki}

\title{Logika i Teoria Mnogości}

\begin{document}

\tableofcontents

\section{Prawa logiki}

\subsection{Prawa łączności}

\begin{itemize}
  \item $(p \land q) \land r \leftrightarrow p \land (q \land r)$
  \item $(p \lor q) \lor r \leftrightarrow p \lor (q \lor r)$
  \item $((p \leftrightarrow q) \leftrightarrow r) \leftrightarrow (p
    \leftrightarrow (q \leftrightarrow r))$
\end{itemize}

\subsection{Prawa przemienności}

\begin{itemize}
  \item $p \land q \leftrightarrow q \land p$
  \item $p \lor q \leftrightarrow q \lor p$
  \item $(p \leftrightarrow q) \leftrightarrow (q \leftrightarrow p)$
\end{itemize}

\subsection{Prawa impotentności}

\begin{itemize}
  \item $p \lor p \leftrightarrow p$
  \item $p \land p \leftrightarrow p$
\end{itemize}

\subsection{Prawo rozdzielności}

$$(p \land q) \lor r \leftrightarrow (p \lor r) \land (q \lor r)$$

\subsection{Prawo de Morgana}

$$
\neg (p \land q) \leftrightarrow \neg p \lor \neg q
$$

\subsection{Prawo podwójnej negacji}

$$
\neg \neg p \leftrightarrow p
$$

\subsection{Prawo transpozycji}

$$
(p \rightarrow q) \leftrightarrow (\neg q \rightarrow \neg p)
$$

\subsection{Prawo eksportacji-importacji}

$$
(p \land q) \rightarrow r \leftrightarrow p \rightarrow (q \rightarrow r)
$$

\section{Wnioskowanie}

$$
\frac{x_1, x_2, \ldots, x_n}{y}
$$

$x_n$ - założenia, $y$ - teza

Wnioskowanie jest dedukcyjne jeżeli $x_1 \land x_2 \dots \land x_n
\rightarrow y$ jest tautologią.

Jeżeli wniosek wynika logicznie z przesłanek to wnioskowanie jest dedukcyjne.

\subsection{Reguły wnioskowania}

Poniższe reguły są zawsze poprawne.

$$
\frac{p, p \rightarrow q}{q}
$$

$$
\frac{p\rightarrow q, q \rightarrow r}{p \rightarrow r}
$$

$$
\frac{p \rightarrow q, \neg q}{\neg p}
$$

$$
\frac{p \rightarrow q, q \rightarrow p}{p \leftrightarrow q}
$$

\section{Przekształcenia}

$$
p \rightarrow q \leftrightarrow \neg p \lor q
$$

$$
(p \leftrightarrow q) \leftrightarrow (p \rightarrow q) \land (q \rightarrow p)
$$

$$
X \downarrow Y = \neg (x \lor y)
$$

$$
X \uparrow Y = \neg (x \land y)
$$

\section{Postaci normalne}

APN - alternatywna postać normalna. Zbiór klauzul nad zmiennymi połączonymi
operatorem alternatywy. $(p \land q) \lor (\neg p \land q)$. Jeśli w APN jest
para klauzul przeciwnych to jest to anty-tautologia.

KPN - koniunkcyjna postać normalna. Zbiór klauzul nad zmiennymi połączonymi
operatorem koniunkcji. $(p \lor q) \land (\neg p \lor q)$. Jeśli w KPN jest
para klauzul przeciwnych to jest to tautologia.

\begin{tabular}{|c|c|c|c|c|c|c|l|l|}
  \hline
  $p$ & $q$ & $r$ & APN & KPN \\ \hline
  1 & 1 & 1 & $p \land q \land r$ & $\neg p \lor \neg q \lor \neg r$ \\ \hline
  1 & 1 & 0 & $p \land q \land \neg r$ & $\neg p \lor \neg q \lor r$ \\ \hline
  1 & 0 & 1 & $p \land \neg q \land r$ & $\neg p \lor q \lor \neg r$ \\ \hline
\end{tabular}

\section{Sekwenty}

Sekwent to para zbiorów formuł logicznych powstający z normalnego zapisu
algebry logicznej.

\begin{tikzpicture}
  \node (a) at (0, 0) {$p \lor q \rightarrow p \land q$};
  \node (b) [below=of a] {$\neg(p \lor q), p \land q$};
  \node (b_1) [below left=of b] {$\neg p, p \land q$};
  \node (b_2) [below right=of b] {$\neg q, p \land q$};
  \node (c) [below left= of b_1] {$\neg p, p$};
  \node (d) [below right= of b_1] {$\neg p, q$};
  \node (e) [below left= of b_2] {$\neg q, p$};
  \node (f) [below right= of b_2] {$\neg q, q$};
  \draw[->] (a) -- (b);
  \draw[->] (b) -- (b_1);
  \draw[->] (b) -- (b_2);
  \draw[->] (b_1) -- (c);
  \draw[->] (b_1) -- (d);
  \draw[->] (b_2) -- (e);
  \draw[->] (b_2) -- (f);
\end{tikzpicture}

Przy pomocy takiego drzewa można sprawdzić czy formula jest tautologią.

\section{Kwantyfikatory}

$$
\forall_{A(x)} B(x) \leftrightarrow \forall_{x} (A(x) \rightarrow B(x))
$$

$$
\exists_{A(x)} B(x) \leftrightarrow \exists_{x} (A(x) \land B(x))
$$

\section{Zbiory}

\begin{itemize}
  \item $X \subset Y \leftrightarrow \forall x (x \in X \rightarrow x \in Y)$
  \item $X \cup Y \leftrightarrow \{x : x \in X \lor x \in Y\}$
  \item $X \cap Y \leftrightarrow \{x : x \in X \land x \in Y\}$
  \item $X \setminus Y \leftrightarrow \{x : x \in X \land \neg x \in Y\}$
  \item $A \div B = \{x : (x \in A \land \neg x \in B) \lor (\neg x
    \in A \land x \in B)\}$
  \item $\bigcup_{i \in I} A_i = \{x: \exists_{i \in I}(x \in A_i)\}$
  \item $\bigcap_{i \in I} A_i = \{x: \forall_{i \in I}(x \in A_i)\}$
\end{itemize}
$\mathbb{U}$ - uniwersum \\
$A' = \mathbb{U} \setminus A$ \\
$A \times B = \{<x, y>: x \in A \land y \in B\}$

\end{document}
