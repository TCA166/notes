\documentclass{../notatki}

\title{Logika i Teoria Mnogości}

\begin{document}

\tableofcontents

\section{Logika}

\subsection{Prawa logiki}

\subsubsection{Prawa łączności}

\begin{itemize}
  \item $(p \land q) \land r \leftrightarrow p \land (q \land r)$
  \item $(p \lor q) \lor r \leftrightarrow p \lor (q \lor r)$
  \item $((p \leftrightarrow q) \leftrightarrow r) \leftrightarrow (p
    \leftrightarrow (q \leftrightarrow r))$
\end{itemize}

\subsubsection{Prawa przemienności}

\begin{itemize}
  \item $p \land q \leftrightarrow q \land p$
  \item $p \lor q \leftrightarrow q \lor p$
  \item $(p \leftrightarrow q) \leftrightarrow (q \leftrightarrow p)$
\end{itemize}

\subsubsection{Prawa impotentności}

\begin{itemize}
  \item $p \lor p \leftrightarrow p$
  \item $p \land p \leftrightarrow p$
\end{itemize}

\subsubsection{Prawo rozdzielności}

$$(p \land q) \lor r \leftrightarrow (p \lor r) \land (q \lor r)$$

\subsubsection{Prawo de Morgana}

$$
\neg (p \land q) \leftrightarrow \neg p \lor \neg q
$$

\subsubsection{Prawo podwójnej negacji}

$$
\neg \neg p \leftrightarrow p
$$

\subsubsection{Prawo transpozycji}

$$
(p \rightarrow q) \leftrightarrow (\neg q \rightarrow \neg p)
$$

\subsubsection{Prawo eksportacji-importacji}

$$
(p \land q) \rightarrow r \leftrightarrow p \rightarrow (q \rightarrow r)
$$

\subsection{Wnioskowanie}

$$
\frac{x_1, x_2, \ldots, x_n}{y}
$$

$x_n$ - założenia, $y$ - teza

Wnioskowanie jest dedukcyjne jeżeli $x_1 \land x_2 \dots \land x_n
\rightarrow y$ jest tautologią.

Jeżeli wniosek wynika logicznie z przesłanek to wnioskowanie jest dedukcyjne.

\subsubsection{Reguły wnioskowania}

Poniższe reguły są zawsze poprawne.

$$
\frac{p, p \rightarrow q}{q}
$$

$$
\frac{p\rightarrow q, q \rightarrow r}{p \rightarrow r}
$$

$$
\frac{p \rightarrow q, \neg q}{\neg p}
$$

$$
\frac{p \rightarrow q, q \rightarrow p}{p \leftrightarrow q}
$$

\subsection{Przekształcenia}

$$
p \rightarrow q \leftrightarrow \neg p \lor q
$$

$$
(p \leftrightarrow q) \leftrightarrow (p \rightarrow q) \land (q \rightarrow p)
$$

$$
X \downarrow Y = \neg (x \lor y)
$$

$$
X \uparrow Y = \neg (x \land y)
$$

\subsection{Postaci normalne}

APN - alternatywna postać normalna. Zbiór klauzul nad zmiennymi połączonymi
operatorem alternatywy. $(p \land q) \lor (\neg p \land q)$. Jeśli w APN jest
para klauzul przeciwnych to jest to anty-tautologia.

KPN - koniunkcyjna postać normalna. Zbiór klauzul nad zmiennymi połączonymi
operatorem koniunkcji. $(p \lor q) \land (\neg p \lor q)$. Jeśli w KPN jest
para klauzul przeciwnych to jest to tautologia.

\begin{tabular}{|c|c|c|c|c|c|c|l|l|}
  \hline
  $p$ & $q$ & $r$ & APN & KPN \\ \hline
  1 & 1 & 1 & $p \land q \land r$ & $\neg p \lor \neg q \lor \neg r$ \\ \hline
  1 & 1 & 0 & $p \land q \land \neg r$ & $\neg p \lor \neg q \lor r$ \\ \hline
  1 & 0 & 1 & $p \land \neg q \land r$ & $\neg p \lor q \lor \neg r$ \\ \hline
\end{tabular}

\subsection{Sekwenty}

Sekwent to para zbiorów formuł logicznych powstający z normalnego zapisu
algebry logicznej.

\begin{tikzpicture}
  \node (a) at (0, 0) {$p \lor q \rightarrow p \land q$};
  \node (b) [below=of a] {$\neg(p \lor q), p \land q$};
  \node (b_1) [below left=of b] {$\neg p, p \land q$};
  \node (b_2) [below right=of b] {$\neg q, p \land q$};
  \node (c) [below left= of b_1] {$\neg p, p$};
  \node (d) [below right= of b_1] {$\neg p, q$};
  \node (e) [below left= of b_2] {$\neg q, p$};
  \node (f) [below right= of b_2] {$\neg q, q$};
  \draw[->] (a) -- (b);
  \draw[->] (b) -- (b_1);
  \draw[->] (b) -- (b_2);
  \draw[->] (b_1) -- (c);
  \draw[->] (b_1) -- (d);
  \draw[->] (b_2) -- (e);
  \draw[->] (b_2) -- (f);
\end{tikzpicture}

Przy pomocy takiego drzewa można sprawdzić czy formula jest tautologią.

\subsection{Kwantyfikatory}

$$
\forall_{A(x)} B(x) \leftrightarrow \forall_{x} (A(x) \rightarrow B(x))
$$

$$
\exists_{A(x)} B(x) \leftrightarrow \exists_{x} (A(x) \land B(x))
$$

\section{Teoria Mnogości}

\subsection{Zbiory}

\begin{itemize}
  \item $X \subset Y \leftrightarrow \forall x (x \in X \rightarrow x \in Y)$
  \item $X \cup Y \leftrightarrow \{x : x \in X \lor x \in Y\}$
  \item $X \cap Y \leftrightarrow \{x : x \in X \land x \in Y\}$
  \item $X \setminus Y \leftrightarrow \{x : x \in X \land \neg x \in Y\}$
  \item $A \div B = \{x : (x \in A \land \neg x \in B) \lor (\neg x
    \in A \land x \in B)\}$
  \item $\bigcup_{i \in I} A_i = \{x: \exists_{i \in I}(x \in A_i)\}$
  \item $\bigcap_{i \in I} A_i = \{x: \forall_{i \in I}(x \in A_i)\}$
\end{itemize}
$\mathbb{U}$ - uniwersum \\
$A' = \mathbb{U} \setminus A$ \\
$A \times B = \{<x, y>: x \in A \land y \in B\}$

\subsection{Relacje}

$$
R^{-1} = \{<y, x>: <x, y> \in R\}
$$

$$
xRy \leftrightarrow <x, y> \in R
$$

\subsubsection{Złożenie relacji}

$$
R \circ S = \{<x, z>: \exists_{y}(xRy \land ySz)\}
$$

\textbf{Przykład:} \\
$R = \{<1, 2>, <2, 3>\}$ \\
$S = \{<2, 3>, <3, 4>\}$ \\
$R \circ S = \{<1, 3>, <2, 4>\}$

\subsubsection{Rodzaje relacji}

\begin{itemize}
  \item Zwrotna $\forall_{x \in A}xRx$
  \item Preciwzwrotna $\neg \exists_{x \in A}xRx$
  \item Symetryczna $\forall_{x, y \in A}(xRy \rightarrow yRx)$
  \item Przeciwsymetryczna $\forall_{x, y \in A}(xRy \rightarrow \neg yRx)$
  \item Antysymetryczna $\forall_{x, y \in A}((xRy \land yRx)
    \rightarrow x = y)$
  \item Przechodnia $\forall_{x, y, z \in A}((xRy \land yRz) \rightarrow xRz)$
  \item Spójna $\forall_{x, y \in A}(xRy \lor yRx)$
  \item Słabospójna $\forall_{x, y \in A}(xRy \lor x = y \lor yRx)$
\end{itemize}

\subsubsection{Relacja równoważności}

Relacja równoważności to relacja która jest zwrotna, symetryczna i przechodnia.
Taka relacja może mieć klasy abstrakcji $[x]_R=\{y : xRy\}$

\subsubsection{Relacje porządkujące}

Relację $R \subset A_2$ nazywamy relacją porządkującą na zbiorze $A_1$ jeżeli:

\begin{itemize}
  \item $R$ jest zwrotna
  \item $R$ jest antysymetryczna
  \item $R$ jest przechodnia
\end{itemize}

Dodatkowo ta relacja może być liniowo porządkująca jeżeli:

\begin{itemize}
  \item $R$ jest spójna
  \item $R$ jest porządkująca
\end{itemize}

Zbiór $X \subset A_1$ jest łańcuchem w zbiorze uporządkowanym $(A_1, R)$ jeżeli
dla dowolnych $x, y \in X$ zachodzi $xRy \lor yRx$.

\subsubsection{Zbiory uporządkowane}

$(X, A)$ to zbiór uporządkowany; $X$ to zbiór, $A$ to relacja porządkująca.

\begin{itemize}
  \item Element najmniejszy a: $\forall_{x \in X}(a \le x)$
  \item Element największy b: $\forall_{x \in X}(x \le b)$
  \item Element minimalny a: $\forall_{x \in X}(x \le a \rightarrow x = a)$
  \item Element maksymalny b: $\forall_{x \in X}(b \le x \rightarrow x = b)$
\end{itemize}

Element największy/najmniejszy jest jedynym elementem maksymalnym/minimalnym,
oraz jest jednocześnie kresem górnym/dolnym.

\subsection{Kresy}

Kres nie musi należeć do zbioru.

\subsubsection{Kres górny}

$$
\alpha = \sup A \leftrightarrow \forall_{x \in A}(x \le \alpha)
$$

\subsubsection{Kres dolny}

$$
\beta = \inf A \leftrightarrow \forall_{x \in A}(\beta \le x)
$$

\subsection{Funkcje}

Relacja binarna $R$ spełniająca prawostronną jednoznaczność to funkcja.

$$
\forall_{x,y,z}(xRy \land xRz \rightarrow y = z)\text{; czyli dla każdego x
jest jedno y}
$$
Przeciwdziedzina funkcji $f: D^*(f) = \{f(x) : x \in D(f)\}$

$$
f: X \rightarrow Y \text{ jeżeli } D(f)=X \land D^*(f) \subset Y
$$
Funkcja odwzorowuje zbiór $X$ na zbiór $Y$ jeżeli $D^*(f)=Y$

$$
\forall_{x_1, x_2 \in X}(f(x_1) = f(x_2) \rightarrow x_1 = x_2)
\rightarrow f\text{ jest iniekcją}
$$
Iniekcja to funkcja różnowarościowa, suriekcja to funkcja
odwzorowująca na zbiór,
a bijekcja to iniekcja i suriekcja. \\
Obraz zbioru to $f[A] = \{f(x) : x \in A\} = \{y: \exists_x(x \in A
\land y = f(x))\}$

$$
f^{-1}[B] = \{x: f(x) \in B\}
$$

\subsection{Liczby naturalne}

$$
S(x) = x \cup \{x\} \text{- następnik zbioru} X
$$

$$
n \in \mathbb{N}, n = \{0, 1, \dots, n-1\}
$$

$$
0 = \emptyset, 1 = \{0\} = S(\emptyset), 2 = \{0, 1\}
$$

$$
\mathbb{N} = \{0, 1, 2, \dots\}
$$
Zbiór $X$ jest indukcyjny jeżeli:

\begin{itemize}
  \item $0 \in X$
  \item $x \in X \rightarrow S(x) \in X$
\end{itemize}
Liczba naturalna to zbiór należący do wszystkich indukcyjnych zbiorów.

\subsection{Zasada indukcji matematycznej}

$$
0 \in X \land \forall_{n \in X}(S(n) \in X) \rightarrow X = \mathbb{N}
$$

$$
f(n) \text{ - formuła} \land \forall_{f(n)}(F(S(n))) \rightarrow
\forall_{n \in \mathbb{N}}(f(n))
$$

\subsection{Liczby całkowite}

$$
\mathbb{Z} = \mathbb{N} \cup \{0\} \cup \{-n: n \in \mathbb{N}\}
$$

\subsection{Liczby wymierne}

$$
\mathbb{Q} = \{x: x = \frac{m}{n}, m \in \mathbb{Z}, n \in \mathbb{N}\}
$$

\subsection{Porządki}

\subsubsection{Porządek produktowy}

$$
a, b \in A \times B \land \forall_{x_A, y_A \in A, x_B, y_B \in
B}(x_A \le_A x_B \land y_A \le_B y_B) \rightarrow a \le b
$$

\subsubsection{Porządek leksykograficzny}

$$
a, b \in A \times B \land \forall_{x_A, y_A \in A, x_B, y_B \in B}(x_A = x_B
\land y_A \le_B y_B) \rightarrow a \le b
$$

\subsection{Właściwe odcinki początkowe}

$(A, \le)$ - zbiór liniowo uporządkowany. Jeżeli $X \subset A$ oraz
$\forall_{x, y \in A}(x \in X \land y < x) \rightarrow y \in X$ to X jest
właściwym odcinkiem początkowym. \\
\textbf{Przykład:}
\begin{itemize}
  \item $A = (\{1, 2, 3, 4, 5\}, \le)$ - zbiór liniowo uporządkowany
  \item $X = \{1, 2, 3\}$ - właściwy odcinek początkowy
  \item $Y = \{2, 3, 4\}$ - nie jest właściwym odcinkiem początkowym
\end{itemize}
\textbf{Przykład:}
\begin{itemize}
  \item $A = (\mathbb{Q}, \le)$ - zbiór liczb wymiernych
  \item $X = \{x \in \mathbb{Q}: x < 0\}$ - właściwy odcinek początkowy
\end{itemize}

$X$ jest właściwym odcinkiem początkowym. Zaczyna się od $-\infty$ i kończy na
0. Jedyne elementy mniejsze od elementów z $X$ to elementy z $X$.

\subsection{Liczby rzeczywiste}

Liczby rzeczywiste definiujemy jako niepuste właściwe odcinki początkowe w
$(\mathbb{Q}, \le)$, nie mające elementu największego.

Zatem liczby wymierne są reprezentowane przez niepuste właściwe odcinki
początkowe w $(\mathbb{Q}, \le)$, które nie mają elementu największego, ale
mają kres górny.

Liczby niewymierne to liczby wymierne, ale nie mają kresu górnego.

\subsection{Dobry porządek}

Porządek $\le$, który ma element najmniejszy w każdym podzbiorze niepustym, jest
dobrym porządkiem. Zbiór dobrze uporządkowany można przedstawić jako serię
mniejszości elementów różnych zbiorów $a_0<a_1<a_2\dots<b_0<b_1\dots$

Przykładem zbioru liniowo uporządkowanego ale nie dobrze uporządkowanego jest
$\mathbb{R}$ lub $\mathbb{Z}$. Każdy podzbiór w $\mathbb{R}$ jest nieskończony,
zatem nie da się skonstruować serii mniejszości.

Funkcja działająca ze zbioru dobrze uporządkowanego $(A, \le)$ do zbioru dobrze
uporządkowanego $(B, \le)$ zachowuje porządek jeśli $\forall_{x \le_A
y; x,y \in A}(f(x) \le_B f(y))$, oraz zachowuje ostry porządek jeśli $\forall_{x
< y; x,y \in A}(f(x) < f(y))$.

\subsection{Homomorfizmy}

Jeżeli funkcja $f: A \rightarrow B$ dla zbiorów liniowo uporządkowanych $(A,
\le_A)$ i $(B, \le_B)$ jest iniekcją i zachowuje porządek to jest to
homomorfizm porządkowy. Jeżeli funkcja jest homomorfizmem porządkowym oraz
bijekcją dla zbiorów uporządkowanych to jest izomorfizmem. Zbiory dla których
istnieje izomorfizm są izomorficzne wobec siebie ($A \backsimeq B$).
Jeżeli $A$ jest
zbiorem dobrze uporządkowanym i funkcja $f$ zachowuje ostry porządek
to $x \le f(x)$ dla każdego $x \in A$. Zbiór dobrze uporządkowany nie
jest izomorficzny z żadnym swoim właściwym odcinkiem początkowym.
Jeżeli $A$ i $B$ są zbiorami dobrze uporządkowanymi i $A \backsimeq
B$ to $A$ i $B$ są izomorficzne z właściwymi odcinkami początkowymi
drugiego zbioru.

\subsection{Aksojmat wyboru}

Dla dowolnej rodziny zbiorów niepustych i parami rozłącznych istnieje zbiór
zawarty w sumie tej rodziny i mający z każdym zbiorem tej rodziny dokładnie
jeden element wspólny.

Ten aksojmat jest równoważny:
\begin{itemize}
  \item Twierdzeniu, że dla dowolnej rodziny zbiorów niepustych
    istnieje funkcja wyboru dla rodziny
  \item Twierdzeniu, że dla każdego zbioru istnieje dobry porządek na
    tym zbiorze
  \item Jeżeli zbiór uporządkowany spełnia warunek łańcucha, to dla
    każdego łańcucha istnieje ograniczenie górne
\end{itemize}

\subsection{Liczby kardynalne}

Zbiór $X$ jest równoliczby ze zbiorem $Y$ jeżeli istnieje bijekcja
$f: X \rightarrow Y$($X \backsim Y$).

$$
\overline{\overline{X}} = |X| = \text{liczba elementów zbioru } X
$$
Zbiór $Y$ jest skończony, jeżeli jest równoliczby z jakąś liczbą naturalną:
$\exists_{x \in \mathbb{N}}(Y \backsim x)$. Zbiór skończony nie jest równoliczny
z żadnym ze swoich podzbiorów.

$$
\aleph_0 = |\mathbb{N}| = \aleph_0 + \aleph_0 = \text{liczba liczb naturalnych}
$$

Moc zbioru to $\aleph_0$ wtedy i tylko wtedy jeżeli X jest zbiorem wszystkich
wyrazów pewnego ciągu nieskończonego bez powtórzeń. Zbiór jest przeliczalny
jeżeli jest skończony lub mocy $\aleph_0$. Zbiór jest niepusty i przeliczalny
wtedy i tylko wtedy gdy jest zbiorem wszystkich wyrazów pewnego ciągu
nieskończonego.\\
\textbf{Zbiór $\mathbb{R}$ jest nieprzeliczalny.}

$$
\mathfrak{c} = |\mathbb{R}| = \mathfrak{c} + \mathfrak{c} > \aleph_0
$$

$$
|X| \le |Y| \leftrightarrow \exists_Z(Z \subset X \land Z \backsim X)
$$

\subsubsection{Twierdzenie Cantora}

$P(X)$ - zbiór potęgowy X

$$
|X| < |P(X)|
$$

\subsection{Twierdzenie Cantora-Bernsteina}

$$
|X| \le |Y| \land |X| \ge |Y| \rightarrow |X| = |Y|
$$

$$
|X| + |Y| = |X \cup Y|
$$

$$
|X| \cdot |Y| = |X \times Y|
$$

\subsection{Twierdzenie Hessenberga}

$$
X \times X \backsim X
$$
dla każdego nieskończonego zbioru $X$

$$
\aleph_{n + 1} > \aleph_n, n \in \mathbb{N}
$$

\subsection{Hipoteza continuum}

$$
\mathfrak{c} = \aleph_1
$$

$$
2^{\aleph_0} = \aleph_1
$$

Udowodniono, że ta hipoteza jest niezależna od aksjomatów teorii mnogości.

\end{document}
