\documentclass{../notatki}

\title{Mechanika Ogólna}

\begin{document}

\tableofcontents

\section{Ściąga z trygonometrii}

$$
\cos \alpha = \frac{x}{r}
$$

$$
\sin \alpha = \frac{y}{r}
$$

\section{Siły}

$$
F = m \cdot a
$$

$$
\vec{F} = m \cdot \vec{a} = (m \cdot a_x, m \cdot a_y, m \cdot a_z) =
\stackrel{\wedge}{F_x} \cdot x + \stackrel{\wedge}{F_y} \cdot y +
\stackrel{\wedge}{F_z} \cdot z
$$
Dla siły grawitacyjnej działającej na ciało o masie $m$ pod kątem
$\alpha$ do osi $x$ mamy:

$$
F_g = m \cdot g \cdot \sin \alpha
$$

$$
F_{avg} = -\frac{\Delta m}{\Delta t} \cdot \Delta v
$$

\section{Ruch prostoliniowy}

$$
a = \frac{\Delta v}{\Delta t} = v'
$$

$$
v = v_0 + at
$$

$$
x - x_0 = \Delta s = v_0t + \frac{1}{2}a t^2
$$

\section{Ruch obrotowy}

Dla promienia $r$, prędkości obiektu $v$ i przyspieszenia dośrodkowego $a$ mamy:

$$
a = \frac{v^2}{r}
$$

\section{Rzut ukośny}

Dla rzutu ukośnego z prędkością początkową $v_0$ i kątem $\alpha$ mamy:

$$
v_{0x} = v_0 \cos \alpha = \text{składnik x prędkości}
$$

$$
v_{0y} = v_0 \sin \alpha = \text{składnik y prędkości}
$$

$$
v_y = v_{0y} - gt = v_0 \sin \alpha - gt = \text{składnik y rzutu}
$$

\section{Praca}

$$
E_K = \frac{mV^2}{2}
$$

$$
W = E_{K1} - E_{K0} = \Delta E_K = F \cdot d = \int F(x) dx
$$

$$
E_p = mgh
$$
Energia potencjalna sprężystości sprężyny:

$$
E_p = \frac{1}{2}kd^2
$$

\section{Pęd}

$$
p = m \cdot v
$$
\textbf{Zasada zachowania pędu:}\\
W układzie izolowanym suma pędów ciał jest stała:

$$
p_{t=0} = p_{t=t}
$$

\subsection{Popęd}

$$
J = F \cdot \Delta t
$$

\section{Obroty}

$$
\omega = \frac{\Delta \phi}{\Delta t}
$$

$$
\alpha = \frac{\Delta \omega}{\Delta t}
$$

$$
\int \alpha dt = \int d\omega
$$

$$
F_c = m \cdot a_c = m \cdot r \cdot \omega^2
$$
gdzie $a_c$ to przyspieszenie dośrodkowe, a $r$ to promień obrotu. Dla obiektu
o długości $l$ i jednorodnym rozłożeniu masy to $r = \frac{l}{2}$.

$$
\text{ rpm} = \frac{2\pi}{60} \text{ rad/s}
$$

\subsection{Moment bezwładności}

$$
I = \sum m_i r_i^2
$$
czyli suma momentów bezwładności wszystkich punktów materialnych w ciele.
Moment bezwładności wyraża opór ciała na zmianę ruchu obrotowego.

Moment obrotowy:
$$
\tau = I \cdot \alpha = I \cdot \frac{\Delta \omega}{\Delta t}
$$

$$
W = \frac{I\omega^2}{2}
$$

\section{Toczenie się}

Energia kinetyczna ruchu postępowego:
$$
E_K = \frac{mV^2}{2}
$$
Energia kinetyczna obrotu:
$$
E_K = \frac{I\omega^2}{2}
$$

\end{document}
