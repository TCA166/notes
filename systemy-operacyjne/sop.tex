\documentclass{../notatki}

\title{Systemy Operacyjne}

\usetikzlibrary{calc}

\begin{document}

\tableofcontents

\section{System Operacyjny}

System operacyjny jest warstwą oprogramowania operującą bezpośrednio
na sprzęcie, której celem jest zarządzanie zasobami systemu
komputerowego i stworzenie użytkownikowi środowiska łatwiejszego do
zrozumienia i wykorzystania.

\subsection{Co robi?}

\begin{itemize}
  \item Pośredniczy między programami a sprzętem
  \item Tworzy środowisko w którym operują programy użytkowników (low
    level interfejsy)
  \item Steruje komputerem i nadzoruje jego działanie
  \item Zarządza dostępem do zasobów
\end{itemize}

\subsection{Jaką rolę pełni system operacyjny?}

\begin{itemize}
  \item System operacyjny dostarcza abstrakcji programom
  \item System operacyjny zarządza wszystkimi składnikami złożonego systemu
  \item System operacyjny zapewnia kontrolowany i uporządkowany
    przydział zasobów
\end{itemize}

\subsection{Zasoby}

\begin{itemize}
  \item Czas procesora
  \item Pamięć operacyjna
  \item System plików
  \item Urządzenia wejścia wyjścia
  \item $\dots$
\end{itemize}

\subsection{Wywołania systemowe}

Wywołania systemowe oddają kontrolę systemowi operacyjnemu za prośbą
procesu i obsługują np.: dostęp do plików, tworzenie wątku itp.

\subsection{Tryb jądra}

Wszystkie współczesne systemy operacyjne rozróżniają tryb jądra i użytkownika.
W trybie jądra operacje oraz dostęp przez procesor do zasobów nie
jest ograniczony.
Jest to w pewnym sensie jak najniższa warstwa jaką może osiągnąć kod.
Aby operować na tej warstwie kod musi być częścią systemu operacyjnego.
W trybie użytkownika dostęp kodu do zasobów jest kontrolowany przez
system operacyjny.

\subsection{Rozruch komputera}

\begin{enumerate}
  \item BIOS weryfikuje urządzenia i ilość pamięci
  \item BIOS określa urządzenie rozruchowe z pamięci CMOS
  \item BIOS wczytuje program rozruchowy
  \item Program rozruchowy wczytuje system operacyjny
  \item System operacyjny odczytuje listę urządzeń z konfiguracji
    BIOS i sterowniki
\end{enumerate}

\section{Procesor}

Współczesne procesory operują zgodnie z architekturą von Neumanna. Ta
architektura zakłada:

\begin{itemize}
  \item Program jak i dane znajdują się w pamięci operacyjnej
  \item Rozkazy wykonuje się dokładnie w tej kolejności w jakiej
    znajdują się w pamięci (odstępstwa: instrukcje skoku, wywołania, powroty)
  \item Procesor przechowuje w rejestrze adres komórki pamięci
    zawierającej kolejną do wykonania instrukcję. W celu pobrania tej
    instrukcji, procesor wystawia odpowiedni adres na magistrali adresowej.
\end{itemize}

\subsection{Cykl rozkazowy}

\begin{enumerate}
  \item Pobranie kodu rozkazu
  \item Pobranie operandów
  \item Składowanie wyniku
  \item Rozpoznania przerwania
\end{enumerate}

\subsection{Przerwania}

\begin{itemize}
  \item Przerwania pozwalają na zatrzymanie cyklu rozkazowego
    procesora, dzięki czemu system operacyjny może odzyskać kontrolę
    lub zareagować na zdarzenie asynchroniczne
  \item Przerwania mogą pochodzić z trzech źródeł:
    \begin{itemize}
      \item Zewnętrzne – I/O, zegary, układy bezpośredniego dostępu do pamięci
      \item Diagnostyczne – w wyniku wykrycia nieoczekiwanego zdarzenia
      \item Programowe – w wyniku wykonania instrukcji przerwania (
        niezależnie od trybu)
    \end{itemize}
  \item Procesor może decydować czy przyjąć nowe przerwanie podczas
    obsługi innego, lub wyłączyć obsługę przerwań zupełnie
  \item Kolejność obsługi przerwań kontroluje kontroler przerwań
\end{itemize}

\subsubsection{Przerwania zegarowe}

Zwykle czasomierz generuje około 100 przerwań na sekundę. Umożliwiają
one systemowi operacyjnemu analizę stanu systemu i reakcję na zmiany

\subsection{Rejestry}

Rejestr to znajdująca się niedaleko procesora pamięć. Procesor wykorzystuje je
do przechowywania danych oraz wyników operacji. W momencie zmiany kontekstu
system operacyjny zachowuje stan rejestrów w pamięci operacyjnej.

\subsection{Wielowątkowość}

Dzięki wielowątkowości procesor może przechować stan procesora, i
potem do niego wrócić. W ten sposób możemy przechowywać stan kilku
procesów (wątków), dając iluzję jednoczesności z wystarczająco
szybkim przełączaniem. Wielowątkowość też pozwala zachować czas
procesora w momentach w których jakiś wątek czeka.

\subsection{Potoki}

Współczesne procesory dzięki specjalnych architekturach mogą
jednocześnie wykonywać etapy cyklu rozkazowego. Np. w trakcie
wykonywania polecenie pobrać kod nowego polecenia

\end{document}
