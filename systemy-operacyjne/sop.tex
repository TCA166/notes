\documentclass{../notatki}

\title{Systemy Operacyjne}

\usetikzlibrary{calc}

\begin{document}

\tableofcontents

\section{System Operacyjny}

System operacyjny jest warstwą oprogramowania operującą bezpośrednio
na sprzęcie, której celem jest zarządzanie zasobami systemu
komputerowego i stworzenie użytkownikowi środowiska łatwiejszego do
zrozumienia i wykorzystania.

\subsection{Co robi?}

\begin{itemize}
  \item Pośredniczy między programami a sprzętem
  \item Tworzy środowisko w którym operują programy użytkowników (low
    level interfejsy)
  \item Steruje komputerem i nadzoruje jego działanie
  \item Zarządza dostępem do zasobów
\end{itemize}

\subsection{Jaką rolę pełni system operacyjny?}

\begin{itemize}
  \item System operacyjny dostarcza abstrakcji programom
  \item System operacyjny zarządza wszystkimi składnikami złożonego systemu
  \item System operacyjny zapewnia kontrolowany i uporządkowany
    przydział zasobów
\end{itemize}

\subsection{Zasoby}

\begin{itemize}
  \item Czas procesora
  \item Pamięć operacyjna
  \item System plików
  \item Urządzenia wejścia wyjścia
  \item $\dots$
\end{itemize}

\subsection{Wywołania systemowe}

Wywołania systemowe oddają kontrolę systemowi operacyjnemu za prośbą
procesu i obsługują np.: dostęp do plików, tworzenie wątku itp.

\subsection{Tryb jądra}

Wszystkie współczesne systemy operacyjne rozróżniają tryb jądra i użytkownika.
W trybie jądra operacje oraz dostęp przez procesor do zasobów nie
jest ograniczony.
Jest to w pewnym sensie jak najniższa warstwa jaką może osiągnąć kod.
Aby operować na tej warstwie kod musi być częścią systemu operacyjnego.
W trybie użytkownika dostęp kodu do zasobów jest kontrolowany przez
system operacyjny.

\subsection{Rozruch komputera}

\begin{enumerate}
  \item BIOS weryfikuje urządzenia i ilość pamięci
  \item BIOS określa urządzenie rozruchowe z pamięci CMOS
  \item BIOS wczytuje program rozruchowy
  \item Program rozruchowy wczytuje system operacyjny
  \item System operacyjny odczytuje listę urządzeń z konfiguracji
    BIOS i sterowniki
\end{enumerate}

\section{Procesor}

Współczesne procesory operują zgodnie z architekturą von Neumanna. Ta
architektura zakłada:

\begin{itemize}
  \item Program jak i dane znajdują się w pamięci operacyjnej
  \item Rozkazy wykonuje się dokładnie w tej kolejności w jakiej
    znajdują się w pamięci (odstępstwa: instrukcje skoku, wywołania, powroty)
  \item Procesor przechowuje w rejestrze adres komórki pamięci
    zawierającej kolejną do wykonania instrukcję. W celu pobrania tej
    instrukcji, procesor wystawia odpowiedni adres na magistrali adresowej.
\end{itemize}

\subsection{Cykl rozkazowy}

\begin{enumerate}
  \item Pobranie kodu rozkazu
  \item Pobranie operandów
  \item Składowanie wyniku
  \item Rozpoznania przerwania
\end{enumerate}

\subsection{Przerwania}

\begin{itemize}
  \item Przerwania pozwalają na zatrzymanie cyklu rozkazowego
    procesora, dzięki czemu system operacyjny może odzyskać kontrolę
    lub zareagować na zdarzenie asynchroniczne
  \item Przerwania mogą pochodzić z trzech źródeł:
    \begin{itemize}
      \item Zewnętrzne – I/O, zegary, układy bezpośredniego dostępu do pamięci
      \item Diagnostyczne – w wyniku wykrycia nieoczekiwanego zdarzenia
      \item Programowe – w wyniku wykonania instrukcji przerwania (
        niezależnie od trybu)
    \end{itemize}
  \item Procesor może decydować czy przyjąć nowe przerwanie podczas
    obsługi innego, lub wyłączyć obsługę przerwań zupełnie
  \item Kolejność obsługi przerwań kontroluje kontroler przerwań
\end{itemize}

\subsubsection{Przerwania zegarowe}

Zwykle czasomierz generuje około 100 przerwań na sekundę. Umożliwiają
one systemowi operacyjnemu analizę stanu systemu i reakcję na zmiany

\subsection{Rejestry}

Rejestr to znajdująca się niedaleko procesora pamięć. Procesor wykorzystuje je
do przechowywania danych oraz wyników operacji. W momencie zmiany kontekstu
system operacyjny zachowuje stan rejestrów w pamięci operacyjnej.

\subsection{Wielowątkowość}

Dzięki wielowątkowości procesor może przechować stan procesora, i
potem do niego wrócić. W ten sposób możemy przechowywać stan kilku
procesów (wątków), dając iluzję jednoczesności z wystarczająco
szybkim przełączaniem. Wielowątkowość też pozwala zachować czas
procesora w momentach w których jakiś wątek czeka.

\subsection{Potoki}

Współczesne procesory dzięki specjalnych architekturach mogą
jednocześnie wykonywać etapy cyklu rozkazowego. Np. w trakcie
wykonywania polecenie pobrać kod nowego polecenia

\section{Urządzenia I/O}

Na urządzenie składa się kontroler i samo urządzenie. System
operacyjny kontaktuje się przy pomocy sterownika z kontrolerem dzięki
czemu ma dostęp do urządzenia. Realizację operacji I/O można przeprowadzić na
trzy sposoby. Procesor może czekać aktywnie, robiąc figuratywnie nic. Lepszym
rozwiązaniem jest wykorzystanie przerwań, gdzie kontroler na końcu
operacji generuje przerwanie. Alternatywnie przy pomocy systemu DMA,
gdzie kontroler zwraca dane do DMA, a DMA po zakończeniu czytania
generuje przerwanie.

\section{Magistrale}

Magistrale w komputerze służą do dystrybucji danych w komputerze. Zazwyczaj
wyróżniamy trzy równoległe kanały:

\begin{itemize}
  \item magistrala adresowa - przekazuje adres komórek z operacjami
  \item magistrala sterująca - przekazuje rodzaj wykonywanej operacji
  \item magistrala danych - przekazuje dane dot. operacji
\end{itemize}

\section{Procesy}

Proces to istota systemu operacyjnego, reprezentująca program wykonywujący się.
Każdy proces ma swoją przestrzeń adresową, czyli wydzielony zakres w pamięci.
Przestrzeń adresowa przechowuje program, dane, zbiór zasobów oraz
blok kontrolny który opisuje stan procesu.

\subsection{Tworzenie procesu}

Na żądanie procesu nowy proces może zostać stworzony. Nowy proces
może współdzielić część zasobów z procesem rodzicem. Usunięcie procesu oznacza
zwolnienie jego zasobów.

\subsection{Wieloprogramowość}

W idealnym świecie żaden z programów nie musiałby czekać na operacje I/O.
Lecz niestety tak nie jest. Obecność innych programów i obsługa innych programów
umożliwia systemowi zmianę kontekstu w momencie blokady I/O.

\subsection{Zarządcy}

Procesy współdzielą zasoby dlatego konieczne jest wdrożenie procedur
zarządzania procesami i zasobami. Wyróżniamy dwa rodzaje zarządców:
zasobów i procesów.

\subsection{Struktury danych}

Koniecznym jest, aby gdzieś zarządcy zapisywali swoje dane. W tym celu istnieje
kilka struktur wykorzystywanych konkretnie do zapisywania wiedzy o
procesie i zasobach.

\begin{itemize}
  \item Deskryptor procesu - używany przez zarządcę procesów w celu
    rejestrowania stanu procesu
  \item Tabela procesów – Zbiór deskryptorów
  \item Deskryptor zasobu – przechowuje informacje o dostępności zasobu
\end{itemize}
W większość systemów wyróżniamy kilka stanów procesu, z reguły przechowywanych
w deskryptorze procesu.

\begin{itemize}
  \item Nowy
  \item Wykonywany
  \item Oczekujący - wykonywanie zostało zakończone, ale wciąż czeka na coś
  \item Gotowy - oczekiwanie zostało zakończone
  \item Zakończony - zabity
  \item Zombie - nie może być zabity, bo ma dzieci
\end{itemize}

\subsection{Przełączenie kontekstu}

Zadania wykonywane są na procesorze w krótkich cyklach z czasem
przypisanym do 1 zadania. Po końcu cyklu następuje przełączenie kontekstu.
Przełączenie kontekstu polega na zachowaniu stanu i kontekstu procesu
i załadowaniu nowego kontekstu. Na kontekst składa się stan tych
zasobów z których proces korzysta.

\subsection{Kolejki procesów}

Procesy są organizowane w kolejki. Wyróżniamy kilka rodzajów tych kolejek.

\begin{itemize}
  \item Kolejka zadań
  \item Kolejka procesów gotowych
  \item Kolejka do urządzenia
  \item Kolejka oczekujących na sygnał synchronizacji
\end{itemize}

\subsection{Planiści}

Planiści to programy szeregujące wybierające procesu z kolejek w taki
sposób aby dążyć do optymalizacji. Większość systemów wyróżnia
następujących planistów:

\begin{itemize}
  \item Planista krótkoterminowy - zajmuje się przydziałem do
    procesora procesów gotowych
  \item Planista średnioterminowy - Zajmuje się wymianą procesów
    pomiędzy pamięcią główną a pamięcią zewnętrzną. Proces w pamięci
    operacyjnej nazywamy aktywnym a ten w przestrzeni wymiany zawieszonym
  \item Planista długoterminowy – Zajmuje się ładowaniem nowych
    programów do pamięci i kontrolą liczby zadań w systemie
\end{itemize}

\subsection{Wątki}

Wątki to lekkie procesy, znajdujące się w dużym procesie. Współdzielą
przestrzeń adresową i większość zasobów. Wątki mogą być realizowane
przez system operacyjny, lub na poziomie procesu, poprzez
przełączanie egzekucji. Obydwa te rozwiązania mają swoje wady i zalety, gdzie
realizacja wątków przez system jest bardziej sprawiedliwe, a przez
proces szybsze.

\section{Przydział procesora}

Procesor to cenny zasób dlatego decyzja który proces będzie
wykonywany następnie jest niezwykle istotna. Planiści podejmują
decyzje, a ekspedytor realizuje ich decyzje i przełącza kontekst.

\subsection{Funkcja priorytetu}

To funkcja zwracająca numeryczny priorytet procesu. Może mieć różne
argumenty w zależności od rodzaju systemu operacyjnego. Argumentami
mogą być np.: czas oczekiwania, czas obsługi, priorytet zewnętrzny etc.

\subsection{Reguła arbitrażu}

Określa co w przypadku remisu. Istnieje wiele podejść do reguły
arbitrażu, szczególnie: losowy
wybór, cykliczne wykonywanie i chronologiczny – wygrywa starszy proces.

\subsection{Tryb decyzji}

Określa czy decyzja jest ostateczna czy nie. Rozróżniamy dwa tryby
decyzji: wywłaszczeniowy lub nie.

\subsection{Ocena algorytmu planowania przez procesor}

\begin{itemize}
  \item Wykorzystanie procesora – procent czasu przez który procesor
    przetwarza dane
  \item Przepustowość – liczba procesów zakończonych w czasie
  \item Sprawiedliwość
  \item Respektowanie priorytetów
  \item Równoważne obciążenie zasobów
\end{itemize}

\subsection{Ocena algorytmu planowania przez użytkownika}

\begin{itemize}
  \item Czas odpowiedzi – czas pomiędzy przedłożeniem zadania a odpowiedzią
  \item Czas opóźnienia – czas od linii krytycznej do momentu
    zakończenia wykonywania
  \item Czas cyklu przetwarzania – ile czasu trwa skończenie konkretnego procesu
  \item Przewidywalność – czas realizacji żądania niezależnie od obciążenia
\end{itemize}

\subsection{Podstawowe algorytmy}

\begin{itemize}
  \item FIFO
  \item LIFO
  \item SJF - shortest job first
  \item RR - round robin - karuzela
  \item SRT - shortest remaining time
\end{itemize}

\section{Pamięć}

Pamięć ma wiele rodzajów i typów. Na podstawie tych charakterystyk
system operuje
na hierarchii pamięci, z najszybszą, a jednocześnie najmniejszą
pamięcią jak najbliżej procesora.

\subsection{Przestrzeń adresowa}

Rozróżniamy przestrzeń fizyczną i logiczną. Procesor i system operuje
na przestrzeni fizycznej. Z kolei system operacyjny wydziela
fragmenty pamięci fizycznej procesom, tworząc im wirtualne pełne
przestrzenie adresowe.

\subsection{Fragmentacja}

W wyniku wielokrotnych operacji na pamięci może dochodzić do
fragmentacji, kiedy pamięć nie jest równomiernie i ciągle
wykorzystywana, nie pozwalając na nowe alokacje pamięci. Fragmentacja
może być wewnętrzna, kiedy proces wykorzystuje więcej pamięci niż
potrzebuje, oraz zewnętrzna, kiedy między blokami są luki.

\subsection{Podział}

Bloki pamięci muszą być jakoś przydzielane procesom. W jaki sposób
zapamiętywać, i dzielić pamięć aby ten proces minimalizował
fragmentację, i aby struktura danych wewnętrzna dot. bloków była
optymalna? Jest kilka głownych strategii:

\begin{itemize}
  \item Podział pamięci na partycje o stałym rozmiarze
  \item Podział pamięci na partycje o różnym rozmiarze
  \item Podział dynamiczny w którym pamięć jest przydzielana z pewną
    dokładnością (np w blokach 1024 B)
  \item Podział na bloki bliźniacze, polegający na połowieniu zbyt
    dużych obszarów pamięci
\end{itemize}

\subsection{Wybór odpowiedniego bloku}

\begin{itemize}
  \item First fit – przydziela się pierwszy znaleziony wystarczająco duży obszar
  \item Best fit – znajduje się najmniejszy pasujący obszar i się
    przydziela procesowi
  \item Next fit – przydziela się pierwsze pasujące miejsce po
    ostatnim przydzieleniu
  \item Worst fit – przydziela się początkowy fragment największego
    wolnego obszaru
\end{itemize}

\subsection{Ładowanie kodu do pamięci}

\begin{itemize}
  \item Ładowanie absolutne – na etapie tworzenia programu
    ładowalnego znane było dokładne miejsce w pamięci; zawiązanie
    adresów fizycznych zostało wykonane na etapie konsolidacji
  \item Ładowanie relokowalne – fizyczna lokalizacja procesu w
    pamięci ustalana jest na etapie ładowania. Konieczne jest
    przeliczenie adresów
  \item Ładowanie dynamiczne w czasie wykonania – adresy są ustalane
    przez MMU przy odwołaniu się do pamięci
\end{itemize}

\subsection{Stronicowanie}

Pamięć fizyczna jest dzielona na ramki o pewnym rozmiarze. Ramkom
fizycznym odpowiadają logiczne strony. Adres jest podzielony na dwie
części: adres strony i przesunięcie.

\subsection{Przechowywanie Tablicy Stron}

Może się okazać że tablica stron jest za duża na ramkę. Dlatego
wprowadza się system hierarchiczny gdzie adres jest dzielony na części.
Pierwsza część adresu adresuje do strony tablic-tablic stron, itd.

\subsection{Segmentacja}

Segmentacja to dynamiczny wariant stronicowania. Obszar procesu jest
dzielony na spójne segmenty o określonym rozmiarze. Rozmiar segmentu
jest dynamiczny. Tak jak przy stronach segmenty też mają swoją tablicę.

\subsection{System hybrydowy}

Aby czerpać zalety stronicowania (małą fragmentacja zewnętrzna i
szybkość działania) i segmentacji (mniejsza fragmentacja wewnętrzna)
można zbudować system hybrydowy. Pamięć składa się z segmentów.
Segmenty te składają się z ramek. W adresie logicznym przesunięcie w
segmencie jest traktowane jako adres ramki. Zamiast adresu bazowego
segmentu jest tablica ramek.

\section{Problemy współbieżności}

Jednoczesne działanie kilku procesów, które mogą chcieć mieć dostęp
do tych samych zasobów, tworzy nową klasę problemów. Można je
rozwiązać przy pomocy następujących narzędzi:

\begin{itemize}
  \item Semafor – int który ogranicza ilość procesów mających dostęp
    do zasobu. 1 zasób 1 semafor
  \item Algorytm Petersona – algorytm zarządzający dwoma procesami
    tak aby tylko jeden mógł mieć dostęp
  \item Monitor – zbiór zmiennych i procedur do których tylko jeden
    proces ma dostęp. Zmienne widzi każdy proces ale procedury tylko
    proces wewnątrz. Kolejnością wejścia zarządza kolejka
\end{itemize}

\section{System plików}

Plik to abstrakcyjny obraz informacji gromadzonej i udostępnianej
przez system komputerowy. System operacyjny ma zapewnić identyfikację
pliku, interfejs do pliku oraz zapewnić bezpieczeństwo i efektywność
operacji na pliku. Struktura fizyczna pliku ma określać jak dane są
przechowywane i jest narzucana przed nośnik. Struktura logiczna pliku
z kolei ma określać informację wewnątrz pliku.

\subsection{Dostęp do pliku}

\begin{itemize}
  \item Sekwencyjny – informacje z pliku są przetwarzane rekord po rekordzie
  \item Bezpośredni – informacje z pliku są przetwarzane w sposób dowolny
  \item Indeksowy – rekordy są identyfikowane przez hash, co
    umożliwia zczytywanie z pliku jak z struktur danych
\end{itemize}

\subsection{Układ systemu plików na dysku}

Dysk składa się z partycji. Na początku dysku jest MBR i tabela
partycji. Potem są partycje.Rekord MBR odpowiada za wyszukanie
aktywnej partycji i odczytanie jej bloku rozruchowego.Za blokiem
rozruchowym jest superblok który odpowiada za najważniejsze parametry
systemu plików.Potem występują informacje o blokach wolnych np.
tabela FAT. Następnie są i-węzły czyli struktury po jednej na plik. I
na koniec tuż przed samymi danymi plików jest opis katalogu root.

\end{document}
