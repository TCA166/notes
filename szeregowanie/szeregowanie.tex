\documentclass{../notatki}

\title{Szeregowanie zadań}

\usetikzlibrary{calc}

\begin{document}

\tableofcontents

\section{Wprowadzenie}

Teoria szeregowania zadań zajmuje się problemami polegającymi na przydzieleniu
pewnych zadań do dostępnych maszyn w taki sposób, aby pewne kryterium było
optymalizowane. Będziemy się zajmować deterministycznymi problemami, czyli
takimi, w których wszystkie dane są znane z góry.

\section{Postać problemu}

Standardowo, problem jest skonstruowany z następujących składowych:

\begin{itemize}
  \item zadania $\mathcal{J} = \{J_1, \dots, J_n\}$
  \item maszyny $\mathcal{P} = \{P_1, \dots, P_m\}$
  \item zasoby $\mathcal{R} = \{R_1, \dots, R_s\}$ dostępnych w $m_1,
    \dots, m_s$ jednostkach
\end{itemize}

\subsection{Maszyny}

W problemie w zależności od wykonywanego zadania i maszyn mogą występować
różne ograniczenia i różnice między maszynami. Jeśli mamy do
czynienia z kilkoma maszynami równoległymi to te maszyny mogą być:
\begin{itemize}
  \item $P$ - identycznościowe czyli z jednakową szybkością
  \item $Q$ - jednorodne czyli z różną szybkością między maszynami
  \item $R$ - dowolne czyli z różniącą się szybkością między
    zadaniami i maszynami
\end{itemize}
Jeśli mamy do czynienia z maszynami dedykowanymi, gdzie każde zadanie składa się
z operacji wykonywanych na różnych maszynach to maszyny mogą być:
\begin{itemize}
  \item $F$ - system przepływowy czyli każde zadanie przechodzi przez
    maszyny w tej samej kolejności
  \item $O$ - system otwarty czyli kolejność wykonywania operacji jest dowolna
  \item $J$ - system gniazdowy czyli każde zadanie ma ustaloną własną
    kolejność przechodzenia przez maszyny
\end{itemize}

\subsection{Zadania}

Zadanie $J$ opisują następujące atrybuty:

\begin{itemize}
  \item $p_j$ - czas wykonania zadania $J_j$
  \item $r_j$ - czas przygotowania zadania $J_j$
  \item $d_j$ - pożądany czas zakończenia zadania $J_j$
  \item $w_j$ - waga zadania $J_j$
\end{itemize}

\subsection{Parametry zadań}

Zbiór zadań $\mathcal{J}$ jako całość opisują ograniczenia
kolejnościowe (acykliczny graf skierowany), oraz
podzielność czyli czy zadania można przerywać i wznawiać.

\subsection{Uszeregowanie}

Uszeregowaniem nazywamy przypisanie każdemu zadaniu maszyny i zasobów w czasie.
Koniecznym jest aby następujące warunki były spełnione:
\begin{itemize}
  \item w każdej chwili maszyna wykonuje tylko jedno zadanie
  \item w każdej chwili każde zadanie jest wykonywane przez jedną maszynę
  \item Każde zadanie jest wykonywane w całości
  \item Spełnione są ograniczenia kolejnościowe
  \item Jeśli zadania są podzielne to są one przerywane skończoną ilość razy
\end{itemize}

\subsubsection{Parametry uszeregowania}

\begin{itemize}
  \item moment rozpoczęcia $S_j$
  \item moment zakończenia $C_j$
  \item czas przepływu $F_j = C_j - S_j$
  \item opóźnienie $L_j = C_j - d_j$
  \item spóźnienie $T_j = \max(0, L_j)$
  \item przyspieszenie $E_j = \max(0, d_j - C_j)$
\end{itemize}

\subsubsection{Kryteria optymalizacji}

Typowo w szeregowaniu optymalizujemy jakąś funkcję składającą się z
parametrów uszeregowania. Przykładowe funkcje to:
$C_{max}=\max(C_j)$, $C_{sum} = \sum C_j$ czy $T_{sum} = \sum T_j$.

\subsection{Notacja Trójpolowa}

$$
\alpha | \beta | \gamma
$$
gdzie $\alpha$ określa ograniczenia maszyn, $\beta$ określa ograniczenia
zadań, a $\gamma$ określa kryterium optymalizacji.

\end{document}